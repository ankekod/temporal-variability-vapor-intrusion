\section{Introduction}\label{s:intro}

Long term vapor intrusion (VI) studies in both residential and larger commercial structures have raised concerns regarding significant observed transient behavior in indoor air contaminant concentrations\cite{u.s._environmental_protection_agency_oswer_2015,folkes_observed_2009,holton_temporal_2013,johnston_spatiotemporal_2014,hosangadi_high-frequency_2017,mchugh_recent_2017,u.s._environmental_protection_agency_assessment_2015}.
Such variations make it difficult for those charged with protecting human health to formulate a response - should evaluation of the risk of exposure be based upon observed peak concentrations, or long-term averages, or something else?
There is even uncertainty within the VI community regarding how to best develop sampling strategies to address this problem\cite{u.s._environmental_protection_agency_oswer_2015,holton_temporal_2013,johnson_integrated_2016}.
What represents a reasonable sampling strategy for a particular site a single 8-hour sample?
Repeated 8-hour samples?
Month-long samples?
Continuous monitoring?\par

VI involves the migration of volatilizing contaminants from soil, groundwater or other subsurface sources into overlying structures.
The basic nature of VI has been understood for some time and it has been the subject of much study, but some aspects remain poorly understood, such as the causes of the sometimes observed large temporal transients in indoor air concentrations.
Results from a house operated by Arizona State University (ASU) near Hill AFB in Utah, an EPA experimental house in Indianapolis, IN and a large warehouse at the Naval Air Station (NAS) North Island, CA have all shown significant transient variations in indoor air contaminant concentrations.
All were outfitted with sampling and monitoring equipment that allowed tracking temporal variation in indoor air contaminant concentrations on time scales of hours.
All have shown that these concentrations vary significantly with time - orders of magnitude on the timescale of a day or days\cite{holton_evaluation_2015,guo_vapor_2015,hosangadi_high-frequency_2017}.\par

In one instance the source of the variation was clearly established during the study of the site.
At the ASU house a drain pipe (or “land drain”) connected to a sewer system was discovered beneath the house.
Careful isolation of this source led to a clear conclusion that this “preferential pathway” for contaminant vapor migration significantly contributed to observed indoor air contaminant levels and their fluctuations\cite{guo_vapor_2015,guo_identification_2015}.
While in this case the issue of a contribution from a preferential pathway was clearly resolved, what it left open was a question of whether existence of such a preferential pathway would always be expected to lead to large fluctuations in indoor air contaminant concentrations.\par

Similarly, a sewer pipe has recently been suggested to be a source of the contaminants found in the EPA Indianapolis house.
That site was also characterized by large indoor air contaminant concentration fluctuations\cite{mchugh_evidence_2017,u.s._environmental_protection_agency_assessment_2015}.
Sewer lines have been previously implicated as VI sources at several sites\cite{pennell_sewer_2013,mchugh_evidence_2017,roghani_occurrence_2018,riis_vapor_2010}.
A Danish study has estimated that roughly 20\% of all VI sites in central Denmark involve significant sewer VI pathways\cite{nielsen_remediation_2017}.
Thus while consideration of sewer or other preferential pathways is now part of normal good practice in VI site investigation\cite{u.s._environmental_protection_agency_oswer_2015}, it is still not known whether the existence of such pathways automatically means that large temporal fluctuations are necessarily to be expected.\par

In some of the above cited cases\cite{pennell_sewer_2013,riis_vapor_2010}, a sewer provided a pathway for direct entry of contaminant into the living space.
While potentially important in many instances, this scenario is not further considered here where the focus is on pathways that deliver contaminant via the soil beneath a structure.
It is, however, now known that even absent a preferential pathway, there may be significant transient variation in indoor air contaminant concentrations at VI sites\cite{folkes_observed_2009,brenner_results_2010,johnston_spatiotemporal_2014}.
One example is the site at NAS North Island at which no preferential pathways have been identified.
Instead, a building at this site is characterized by significant temporal variations in indoor-outdoor pressure differential\cite{hosangadi_high-frequency_2017}.
It is believed that this is the origin of the observed indoor air contaminant concentration fluctuations at that site.\par

This paper investigates the sources of the temporal variation in indoor air contaminant concentrations in both the presence and absence of preferential pathways.
In this work, the latter scenarios are referred to as “normal” VI scenarios, in which there is typically a groundwater source of the contaminant.
Specifically, we pose the question of just how much variation in indoor air contaminant concentration may be expected at such normal VI sites vs. those characterized by preferential pathways within the soil beneath the site.
The conditions required for preferential pathways to become significant contributors to temporal variations in indoor air contaminant concentrations are also explored, and the consequences for sampling strategies are discussed.\par
