\documentclass[preprint,12pt]{elsarticle}

% additional packages
\usepackage[utf8]{inputenc}
%\bibliographystyle{abbrvnat}
\usepackage{amsmath}
\usepackage{booktabs}
\usepackage{subcaption}
\usepackage{tabularx}
\usepackage{array,multirow,multicol,graphicx}
\usepackage{comment}
\graphicspath{
  {./../figures/},
}
\usepackage{lineno}
\linenumbers
% macros

\journal{Building and Environment}

\begin{document}

\begin{frontmatter}
% title
\title{Factors Affecting Temporal Variations In Vapor Intrusion-Induced Indoor Air Contaminant Concentrations}

% authors
\author[Brown]{Jonathan G. V. Ström}
\author[ASU]{Yuanming Guo}
\author[ASU]{Yijun Yao}
\author[Brown]{Eric M. Suuberg}

\ead{eric\_suuberg@brown.edu}

\address[Brown]{Brown University, School of Engineering, Providence, RI, USA}
\address[ASU]{Arizona State University, School of Sustainable Engineering and the Building Environment, Tempe, AZ, USA}

\begin{abstract}
Temporal variability in indoor air contaminant concentrations at vapor intrusion (VI) sites has been a concern for some time.
We consider the source of the reported variability at VI sites located near Hill Air Force Base (AFB) in Utah, an EPA experimental house in Indiana, and Naval Air Station North Island in California.
We focus in particular on how the indoor/outdoor pressure differences and air exchange rates affected indoor air contaminant concentrations at these sites.
We investigate how these dynamics differ for a site that is characterized by a preferential pathway (like Hill AFB) and VI sites that are not influenced by such pathways, using three-dimensional fluid dynamics models and statistical analysis of the aforementioned sites.
A preferential pathway can dramatically increase a VI site's sensitivity to build pressurization, provided there exist a medium allowing effective communication between a contaminant-delivering preferential pathway and the indoor air space, e.g. a permeable subslab space that may be provided by a gravel layer.
Preferential pathways may also erroneously indicate the presence of indoor contaminant sources.
At sites characterized by significant advective transport from the subslab to the indoor air space, much of the short-term variability in indoor air contaminant concentration can be explained by an impact of fluctuations in indoor/outdoor pressure differences.
Meanwhile, air exchange rate variation drives most of the short-term variability at sites characterized by minor variations in advective transport.
\end{abstract}

\begin{keyword}
  Vapor intrusion \sep Preferential pathways \sep Temporal variability \sep Attenuation factor \sep Air exchange rate \sep Indoor/outdoor pressure difference
\end{keyword}

\end{frontmatter}

\section{Introduction}\label{s:intro}

Long term vapor intrusion (VI) studies in both residential and larger commercial structures have raised concerns regarding significant observed transient behavior in indoor air contaminant concentrations\cite{u.s._environmental_protection_agency_oswer_2015,folkes_observed_2009,holton_temporal_2013,johnston_spatiotemporal_2014,hosangadi_high-frequency_2017,mchugh_recent_2017,u.s._environmental_protection_agency_assessment_2015}.
Such variations make it difficult for those charged with protecting human health to formulate a response - should evaluation of the risk of exposure be based upon observed peak concentrations, or long-term averages, or something else?
There is even uncertainty within the VI community regarding how to best develop sampling strategies to address this problem\cite{u.s._environmental_protection_agency_oswer_2015,holton_temporal_2013,johnson_integrated_2016}.
What represents a reasonable sampling strategy for a particular site a single 8-hour sample?
Repeated 8-hour samples?
Month-long samples?
Continuous monitoring?\par

VI involves the migration of volatilizing contaminants from soil, groundwater or other subsurface sources into overlying structures.
The basic nature of VI has been understood for some time and it has been the subject of much study, but some aspects remain poorly understood, such as the causes of the sometimes observed large temporal transients in indoor air concentrations.
Results from a house operated by Arizona State University (ASU) near Hill AFB in Utah, an EPA experimental house in Indianapolis, IN and a large warehouse at the Naval Air Station (NAS) North Island, CA have all shown significant transient variations in indoor air contaminant concentrations.
All were outfitted with sampling and monitoring equipment that allowed tracking temporal variation in indoor air contaminant concentrations on time scales of hours.
All have shown that these concentrations vary significantly with time - orders of magnitude on the timescale of a day or days\cite{holton_evaluation_2015,guo_vapor_2015,hosangadi_high-frequency_2017}.\par

In one instance the source of the variation was clearly established during the study of the site.
At the ASU house a drain pipe (or “land drain”) connected to a sewer system was discovered beneath the house.
Careful isolation of this source led to a clear conclusion that this “preferential pathway” for contaminant vapor migration significantly contributed to observed indoor air contaminant levels and their fluctuations\cite{guo_vapor_2015,guo_identification_2015}.
While in this case the issue of a contribution from a preferential pathway was clearly resolved, what it left open was a question of whether existence of such a preferential pathway would always be expected to lead to large fluctuations in indoor air contaminant concentrations.\par

Similarly, a sewer pipe has recently been suggested to be a source of the contaminants found in the EPA Indianapolis house.
That site was also characterized by large indoor air contaminant concentration fluctuations\cite{mchugh_evidence_2017,u.s._environmental_protection_agency_assessment_2015}.
Sewer lines have been previously implicated as VI sources at several sites\cite{pennell_sewer_2013,mchugh_evidence_2017,roghani_occurrence_2018,riis_vapor_2010}.
A Danish study has estimated that roughly 20\% of all VI sites in central Denmark involve significant sewer VI pathways\cite{nielsen_remediation_2017}.
Thus while consideration of sewer or other preferential pathways is now part of normal good practice in VI site investigation\cite{u.s._environmental_protection_agency_oswer_2015}, it is still not known whether the existence of such pathways automatically means that large temporal fluctuations are necessarily to be expected.\par

In some of the above cited cases\cite{pennell_sewer_2013,riis_vapor_2010}, a sewer provided a pathway for direct entry of contaminant into the living space.
While potentially important in many instances, this scenario is not further considered here where the focus is on pathways that deliver contaminant via the soil beneath a structure.
It is, however, now known that even absent a preferential pathway, there may be significant transient variation in indoor air contaminant concentrations at VI sites\cite{folkes_observed_2009,brenner_results_2010,johnston_spatiotemporal_2014}.
One example is the site at NAS North Island at which no preferential pathways have been identified.
Instead, a building at this site is characterized by significant temporal variations in indoor-outdoor pressure differential\cite{hosangadi_high-frequency_2017}.
It is believed that this is the origin of the observed indoor air contaminant concentration fluctuations at that site.\par

This paper investigates the sources of the temporal variation in indoor air contaminant concentrations in both the presence and absence of preferential pathways.
In this work, the latter scenarios are referred to as “normal” VI scenarios, in which there is typically a groundwater source of the contaminant.
Specifically, we pose the question of just how much variation in indoor air contaminant concentration may be expected at such normal VI sites vs. those characterized by preferential pathways within the soil beneath the site.
The conditions required for preferential pathways to become significant contributors to temporal variations in indoor air contaminant concentrations are also explored, and the consequences for sampling strategies are discussed.\par

\section{Methods}\label{s:methods}

\subsection{Statistical Analysis Of Field Data}\label{s:methods_stats}

To frame the question of just how much variability in indoor air contaminant concentrations is actually observed, field datasets have been analyzed.
For this purpose, datasets from the ASU house in Utah, the EPA Indianapolis site and North Island NAS were chosen for analysis.
Readers are referred to the original published works for details regarding data acquisition\cite{holton_evaluation_2015,guo_vapor_2015,holton_temporal_2013,hosangadi_high-frequency_2017,u.s._environmental_protection_agency_assessment_2015}.\par

The ASU house data were obtained over a period of several years.
During part of this time, controlled pressure method (CPM) tests were being conducted, in which the house was underpressurized to an extent greater than that characterizing “normal” house operation: increasing VI potential\cite{mchugh_evaluation_2012,mchugh_recent_2017,holton_evaluation_2015}.
The period of CPM testing is thus excluded from the analysis.
Likewise, the existence of a preferential pathway at the ASU house needs to be considered in examining that dataset; during some of the testing at that site, this pathway was cut off, resulting in “normal” VI conditions in which the main source of contaminant was diffusion of contaminant vapor from an underlying groundwater source.\par

The NAS North Island dataset has not (as far as is known) been influenced by a preferential pathway, but the structure there was subject to “large” internal pressure fluctuations.
By “large” is meant still only of order 10-20 Pa, but these were greater than those generally recorded at the ASU house during normal operations.
The underlying soil at NAS North Island is sandy\cite{hosangadi_high-frequency_2017} and more permeable than that at the ASU site, which will be shown to lead to greater pressure sensitivity in the former case.\par

The Indianapolis site investigation also spanned a number of years and periodically included the testing of a sub-slab depressurziation system (SSD) for VI mitigation.
Only the period before the installation of this system was considered in the present analysis.
It is likely a sewer line beneath the structure acted as a preferential pathway\cite{mchugh_evidence_2017}.
Unlike at the ASU house, this preferential pathway was never removed or blocked, making it impossible to isolate the role of the preferential pathway at this site.
It is still of interest to consider the data from this site because of the completeness and extensiveness of the data collection.
Figure \ref{fig:indianapolis} illustrates a typical reported series of indoor air trichloroethylene (TCE) concentration measurements from this site.
There is almost a two order of magnitude variation in the concentration data.\par

\begin{figure}[htb!]
  \centering
 \includegraphics[width=\textwidth]{time_indianapolis.pdf}
 \caption{Typical data on indoor air TCE contaminant concentrations at the Indianapolis site\cite{u.s._environmental_protection_agency_assessment_2015}.}\label{fig:indianapolis}
\end{figure}

Some of the analysis of the above three field data sets relies on a probability density estimation technique called ”kernel density estimation” (KDE).
KDE is a technique used for estimating the probability distribution of a random variable(s) by using multiple kernels, or weighting functions to characterize the data sets.
In this case, Gaussian kernels are used to create the KDEs.
This means that it is presumed that the variables of interest (i.e., indoor air contaminant concentrations and indoor-outdoor pressure differentials, as sampled) are normally distributed around mean values and that there are statistical fluctuations associated with each sampling event.
In this instance, the scipy statistical package was used to construct the KDEs, assuming a bandwidth parameter determined by Scott’s rue.
The SciPy Python library was used to conduct all statistical analysis and data processing\cite{jones_scipy_2011}.\par

\subsection{Modeling Work}\label{s:methods_model}

A previously described three-dimensional computational fluid dynamics model of a generic VI impacted house has been used to elucidate certain aspects of transient VI processes.
In the present work, there has been an addition of a preferential pathway to the “standard” model that has been described before in publications by this group\cite{shen_influence_2013,yao_investigating_2017,yao_three-dimensional_2017}.
As in the earlier studies, only the vadose zone soil domain is directly modeled.
Figure \ref{fig:model} shows a cutaway view of the relevant modeling domain.\par

\begin{figure}[htb!]
  \centering
 \includegraphics[width=\textwidth]{model.png}
 \caption{Foundation and vadose zone soil represented in the modeling. Note that here a gravel sub-base material is shown, but in certain simulations, that material is absent and the surrounding soil directly contacts the foundation slab.  Different assumptions are made regarding the preferential pathway, here shown as a pipe entering the gravel sub-base. In some cases, the preferential pathway has been "turned off".}\label{fig:model}
\end{figure}

The modeled VI impacted structure is assumed to have a 10x10 m foundation footprint, with the bottom of the foundation slab lying 1 m below ground surface (bgs), simulating a house with a basement.
The indoor air space is modeled as a continuously stirred tank (CST)\cite{u.s._environmental_protection_agency_oswer_2015} and all of the contaminant entering the house is assumed to enter with soil gas through a 1 cm wide crack located between the foundation walls and the foundation slab around the  perimeter of the house.
All of the contaminant leaving the indoor air space is assumed to do so via air exchange with the ambient.
The indoor control volume is here assumed to consist of only of the basement, having a total volume of $300 \; \mathrm{m^3}$.
Clearly different assumptions could be made regarding the structural features and the size of the crack entry route, but for present purposes, this is unimportant as the intent is only to show for “typical” values what the influence of some critical parameters is.\par

The modeled surrounding soil domain extends 5 meters from the perimeter of the house and is assumed to consist of sandy loam, except as noted otherwise.
Directly beneath the foundation slab, there is assumed to be a 30 cm (one foot) thick gravel layer, except in certain cases here this sub-base material is assumed to be the same as the surrounding soil (termed a "uniform" soil scenario).
The groundwater beneath the structure is assumed to be homogeneously contaminated with TCE selected as a prototypical contaminant.
The groundwater itself is not modeled, as the bottom of the model domain is defined by the top of the water table.
Where relevant, the preferential pathway is modeled as a 10 cm (4”) pipe that opens into the gravel sub-base beneath the structure.
The air in the pipe is also assumed to be contaminated with TCE at a vapor concentration equal to the vapor in equilibrium with the groundwater contaminant concentration below the structure, modified by a scaling factor $\chi$ (allowing the contaminant concentration in the pipe to be parameterized).
This model illustrates the concept of a "preferential pathway", as the pipe carries contaminant vapor to the immediate vicinity of the foundation, by a path that circumvents the usual soil diffusion pathway.\par

The ground surface and the pipe are both sources of air to the soil domain.
Both are assumed to exist at reference atmospheric pressure.
Soil gas transport is governed by Richard’s equation, a modified version of Darcy’s Law, taking the variability of soil moisture in the vadose zone into account\cite{richards_capillary_1931}.
The van Genuchten equations are used to predict the soil moisture content and thus the effective permeability of the soil\cite{van_genuchten_closed-form_1980}.
The effective diffusivity of contaminant in soil is calculated using the Millington-Quirk model\cite{millington_permeability_1961}.
The transport of contaminant vapor in the soil is assumed to be governed by the advection-diffusion equation, in which either advection or diffusion may dominate depending upon position and particular circumstances.
The key working equations and the boundary conditions are summarized in Table \ref{tbl:eqns_bc_parameters}.\par

\begin{table}[htb!]
  \setlength{\tabcolsep}{1pt}
  \centering
  \begin{tabular}{l c c c c c c}
    %%%%%%%%%%%%%%%%%%%%%%%%%%%%%%%%%%%%%%%%%%%%%%%%%%%%%%%%%%%%%%%%%%%%%%%%%%%%
    % GOVERNING EQUATIONS
    %%%%%%%%%%%%%%%%%%%%%%%%%%%%%%%%%%%%%%%%%%%%%%%%%%%%%%%%%%%%%%%%%%%%%%%%%%%%
    \toprule
    \multicolumn{7}{c}{\textbf{Governing Equations}} \\
    \midrule
    % CSTR
    Unsteady-CST & \multicolumn{6}{c}{$V\frac{d c_\mathrm{in}}{d t} = \int_{A_\mathrm{ck}} j_\mathrm{ck} dA - c_\mathrm{in} A_e V_\mathrm{slab}$} \\
    % RICHARDS
    Richard's & \multicolumn{6}{c}{$\nabla \cdot \rho \Big( - \frac{\kappa_s}{\mu} k_r \nabla p \Big) = 0$} \\
    % TRANSPORT
    Transport & \multicolumn{6}{c}{$\frac{\partial}{\partial t} \Big( \theta_w c_w + \theta_g c \Big) = \nabla (D_\mathrm{eff} \cdot \nabla c) - \vec{u} \cdot \nabla c$} \\
    % MILLINGTON-QUIRK
    Millington-Quirk & \multicolumn{6}{c}{$D_\mathrm{eff} = D_\mathrm{air}\frac{\theta_g^{10/3}}{\theta_t^2} + \frac{D_\mathrm{water}}{K_H} \frac{\theta_w^{10/3}}{\theta_t^2}$} \\
    % VAN GENUCHTEN
    \multirow{4}{*}{van Genuchten} & \multicolumn{6}{c}{$\mathrm{Se} = \frac{\theta_w - \theta_r}{\theta_t - \theta_r} = [1 + |\alpha z|^n]^{-m}$} \\
     & \multicolumn{6}{c}{$\theta_g = \theta_t - \theta_w$}\\
     & \multicolumn{6}{c}{$k_r = (1-\mathrm{Se})^{l}[1(\mathrm{Se}^{m})^m]^2$} \\
     & \multicolumn{6}{c}{$m = 1 - 1/n$} \\
     %%%%%%%%%%%%%%%%%%%%%%%%%%%%%%%%%%%%%%%%%%%%%%%%%%%%%%%%%%%%%%%%%%%%%%%%%%%%
     % BOUNDARY CONDITIONS
     %%%%%%%%%%%%%%%%%%%%%%%%%%%%%%%%%%%%%%%%%%%%%%%%%%%%%%%%%%%%%%%%%%%%%%%%%%%%
    \midrule
    \multicolumn{7}{c}{\textbf{Boundary Conditions}} \\
    \midrule
    \textbf{Boundary} & \multicolumn{3}{c}{\textbf{Richard's Eqn.}} & \multicolumn{3}{c}{\textbf{Transport Eqn.}} \\
    % Foundation crack
    Foundation crack & \multicolumn{3}{c}{$p = p_\mathrm{in/out} \; \mathrm{(Pa)}$} & \multicolumn{3}{c}{$j_\mathrm{ck} = \frac{u c}{1 - \exp{(u L_\mathrm{slab}/D_\mathrm{air})}}$} \\
    % Groundwater source
    Groundwater & \multicolumn{3}{c}{\textit{No flow}} & \multicolumn{3}{c}{$c = c_\mathrm{gw} K_H \; \mathrm{(\mu g/m^3)}$} \\
    % Ground surface
    Ground surface & \multicolumn{3}{c}{$p = 0 \; \mathrm{(Pa)}$} & \multicolumn{3}{c}{$c = 0 \; \mathrm{(\mu g/m^3)}$} \\
    % Preferential pathway
    \multirow{2}{*}{\begin{minipage}{0.2\textwidth}Preferential\\pathway\end{minipage}} & \multicolumn{3}{c}{\multirow{2}{*}{$p = 0 \; \mathrm{(Pa)}$}} & \multicolumn{3}{c}{\multirow{2}{*}{$c = c_\mathrm{gw} K_H \chi \; \mathrm{(\mu g/m^3)}$}} \\ \\
    %%%%%%%%%%%%%%%%%%%%%%%%%%%%%%%%%%%%%%%%%%%%%%%%%%%%%%%%%%%%%%%%%%%%%%%%%%%%
    % SOIL PROPERTIES
    %%%%%%%%%%%%%%%%%%%%%%%%%%%%%%%%%%%%%%%%%%%%%%%%%%%%%%%%%%%%%%%%%%%%%%%%%%%%
    \midrule
    \multicolumn{7}{c}{\textbf{Soil Properties}\cite{dan_capillary_2012,abreu_conceptual_2012,u.s._environmental_protection_agency_userss_2004}} \\
    \midrule
    \textbf{Soil} & \textbf{$\kappa_s \; \mathrm{(m^2)}$} & \textbf{$\theta_s$} & \textbf{$\theta_r$} & \textbf{$\alpha \; \mathrm{(1/m)}$} & \textbf{$n$} \\
    % GRAVEL
    Gravel & $1.3 \cdot 10^{-9}$ & 0.42 & 0.005 & 100 & 3.1 \\
    % SANDY LOAM
    Sandy Loam & $5.9 \cdot 10^{-13}$ & 0.39 & 0.039 & 2.7 & 1.4 \\
    %%%%%%%%%%%%%%%%%%%%%%%%%%%%%%%%%%%%%%%%%%%%%%%%%%%%%%%%%%%%%%%%%%%%%%%%%%%%
    % TCE PROPERTIES
    %%%%%%%%%%%%%%%%%%%%%%%%%%%%%%%%%%%%%%%%%%%%%%%%%%%%%%%%%%%%%%%%%%%%%%%%%%%%
    \midrule
    \multicolumn{7}{c}{\textbf{Trichloroethylene (diluted in air) Properties}\cite{abreu_conceptual_2012,u.s._environmental_protection_agency_userss_2004}} \\
    \midrule
     & \multicolumn{1}{c}{\textbf{$D_\mathrm{air} \; \mathrm{(m^2/h)}$}} & \textbf{$D_\mathrm{water} \; \mathrm{(m^2/h)}$} & \textbf{$\rho \; \mathrm{(kg/m^3)}$} & \textbf{$\mu \; \mathrm{(Pa \cdot s)}$} & \textbf{$K_H$} \\
     & \multicolumn{1}{c}{$2.47 \cdot 10^{-2}$} & $3.67 \cdot 10^{-6}$ & 1.614 & $1.86 \cdot 10^{-5}$ & 0.403 \\
    %%%%%%%%%%%%%%%%%%%%%%%%%%%%%%%%%%%%%%%%%%%%%%%%%%%%%%%%%%%%%%%%%%%%%%%%%%%%
    % BUILDING PROPERTIES
    %%%%%%%%%%%%%%%%%%%%%%%%%%%%%%%%%%%%%%%%%%%%%%%%%%%%%%%%%%%%%%%%%%%%%%%%%%%%
    \midrule
    \multicolumn{7}{c}{\textbf{Building Properties}} \\
    \midrule
     & \multicolumn{1}{c}{\textbf{$V_\mathrm{base} \; \mathrm{(m^3)}$}} & \textbf{$L_\mathrm{slab} \; \mathrm{(cm)}$} & \textbf{$A_e \; \mathrm{(1/hr)}$} \\
     & \multicolumn{1}{c}{300} & 15 & 0.5 \\
    \bottomrule
  \end{tabular}
  \caption{Governing equations, boundary conditions \& model input parameters. See Table \ref{tbl:abbreviations} for nomenclature.}\label{tbl:eqns_bc_parameters}
\end{table}

\newpage
\input{results.tex}
\section{Conclusions}

Based on statistical analysis of field data from vapor intrusion sites showing significant temporal variations in indoor air contaminant concentrations, supplemented by computational fluid dynamics modeling, it is concluded that several different factors can play a role in determining these variations.
There can be a significant role of advective transport in determining observed variability.
For advective transport to play a decisive role, the present results show that there must exist a sub-foundation zone that permits significant flow of soil gas.
In the case of NAS North Island, a relatively permeable sub-foundation soil permitted fairly high variations in building indoor pressure to cause significant fluctuations in contaminant entry rate.
In the case of the ASU house, a permeable gravel sub-base allowed good communication of the indoor air environment with the contents of a preferential contaminant pathway that entered that sub-base, even though the pressure driving force was not nearly as large as that at NAS North Island.
Both situations led to well-documented large variations (two or three orders of magnitude) in indoor air contaminant concentrations.\par

These findings, as well as those from others reaffirm that screening for preferential pathways should be a routine part of VI investigations.
Sewers have been shown in several studies to play significant roles as potential preferential pathways.
Thus, sampling nearby manholes for the presence for contaminants appears prudent in terms of establishing whether such a preferential pathway may play a role at a site.
This is consistent with what has been suggested by Nielsen et al.\cite{nielsen_remediation_2017}, Pennell et al.\cite{pennell_sewer_2013}, and McHugh et al.\cite{mchugh_evidence_2017}.\par

Even in the absence of a preferential pathway, the present results suggest that some temporal monitoring of indoor/outdoor pressure differences can provide insights regarding whether an unusually large advective driving force for contaminant entry exists.
As was observed at NAS North Island, significant fluctuations in pressure driving force was what lead to significant temporal fluctuations in indoor air contaminant concentrations.
Generally speaking, larger pressure fluctuations on sites with relatively permeable soils and/or leaky buildings would be expected to lead to more significant short-term temporal variability in indoor air contaminant concentrations, suggesting monitoring that can track such variations.\par

Modeling work has shown that the existence of preferential pathways that transport contaminant to a permeable sub-slab zone can lead to significant spatial variability in contaminant concentration in the subslab, even if the subslab is reasonably permeable (such as a gravel sub-base).
The presumption of a well-mixed sub-base is unlikely to be appropriate in the presence of a preferential pathway into the sub-base.
This warns that taking spatially limited subslab vapor samples may provide a misleading picture of the potential for contaminant entry.
Such limited sampling in such situations could even lead to apparent to apparent subslab to indoor air attenuation factors exceeding the typical EPA recommended value of 0.03 (and they can even exceed unity, potentially leading to an erroneous conclusion that an indoor source is present).
This is in addition to other warnings that have been previously presented regarding over-reliance on subslab samples in VI investigations\cite{chow_concentration_2007,folkes_observed_2009,u.s._environmental_protection_agency_assessment_2015,pennell_development_2009}.
The fact that the building itself can influence the subslab contaminant concentrations has also been the subject of caution\cite{holton_creation_2018}.
Thus, the list of reasons for exercising caution in relying upon subslab vapor measurements continues to grow.\par

In the absence of a preferential pathway (a situation that existed at the ASU house when the pathway was closed off) and in the presence of the observed modest stack effect pressure driving force, a temporal variation in indoor air contaminant concentration of roughly an order of magnitude was still observed.
Statistical analysis of that dataset showed, however, that indoor air contaminant concentrations were actually unlikely to vary by more than a factor of three over a week-long or shorter sampling period.
The small variability that was seen over these short times could be explained by the inherent variability in air exchange rates.
It was when indoor air samples taken over longer periods were compared that an almost order of magnitude variation became apparent.
These results were consistent with seasonal timescale variations in the intrusion processes.
The above results suggest that 24-hour indoor air sampling may be appropriate for capturing the variability that actually exists over the timescale of days. Considering sampling over longer time-periods, i.e. 48- or 72-hours, the results seemed to indicate that they would not give different concentration than a 24-hour sample.
For the same reason, multiple rounds of sampling over the time period of a few days or weeks cannot normally be expected to reveal significant differences, unless one is dealing with a preferential pathway situation which can lead to variations on shorter timescales.\par

The analysis in Figure \ref{fig:resampling} also implies that if a 24-hour sample gives an indoor air contaminant concentration that is an order of magnitude lower than a relevant regulatory limit, it is unlikely that the indoor air contaminant concentration will exceed this value at a later date unless there exists a preferential pathway whose contribution varies significantly with time.\par


\section*{Acknowledgements}
This project was supported by grant ES-201502 from the Strategic Environmental Research and Development Program and Environmental Security Technology Certification Program (SERDP-ESTCP).\par

Declaration of interest: none

\clearpage
\input{appendix.tex}

\bibliographystyle{elsarticle-num}
\bibliography{library}

\end{document}
