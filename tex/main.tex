\documentclass[journal=esthag,manuscript=article]{achemso}

% additional packages
\usepackage[utf8]{inputenc}
\usepackage{amsmath}
\usepackage{booktabs}
\usepackage{subcaption}
\usepackage{tabularx}
\usepackage{array,multirow,graphicx}
\usepackage{comment}
\graphicspath{
  {./../figures/},
  {./../figures/air-exchange-rate/},
  {./../figures/kde-iacc-pressure/},
  {./../figures/simulation_predictions/},
  {./../figures/pair_grids/},
  {./../figures/multivariate_analysis/}
}
\usepackage{lineno}
\linenumbers
% macros

% authors
\author{Jonathan G. V. Ström}
\affiliation[Brown University]{Brown University, School of Engineering, Providence, RI, USA}
\author{Yijun Yao}
\affiliation[Zhejiang University]{Zhejiang University, Hangzhou, China}
\author{Eric M. Suuberg}
\email{eric_suuberg@brown.edu}
\affiliation[Brown University]{Brown University, School of Engineering, Providence, RI, USA}

% title
\title{Transient Variability In Vapor Intrusion And The Factors That Influence It}

% keywords
\abbreviations{VI}
\keywords{Vapor intrusion, Preferential pathways, Temporal variability, Factor analysis, Modeling}

\begin{document}

\begin{abstract}

\end{abstract}

\section{Introduction}

Long term vapor intrusion (VI) studies in both residential and larger commercial structures have raised concerns regarding significant observed transient behavior in indoor air contaminant concentrations\cite{u.s._environmental_protection_agency_oswer_2015,folkes_observed_2009,holton_temporal_2013,johnston_spatiotemporal_2014,hosangadi_high-frequency_2017,mchugh_recent_2017,u.s._environmental_protection_agency_assessment_2015}.
VI involves the migration of volatilizing contaminants from soil, groundwater or other subsurface sources into overlying structures. VI has been a recognized problem for some time, but many aspects remain poorly understood, particularly with respect to the causes of large temporal transients in indoor air concentrations.
There is uncertainty within the VI community regarding how to best develop sampling strategies to address this problem\cite{u.s._environmental_protection_agency_oswer_2015,holton_temporal_2013,johnson_integrated_2016}. \par

Results from a house operated by Arizona State University (ASU) near Hill AFB in Utah, an EPA experimental house in Indianapolis, IN and a large warehouse at the Naval Air Station (NAS) North Island, CA have all shown significant transient variations in indoor air contaminant concentrations.
All were outfitted with sampling and monitoring equipment that allowed tracking temporal variation in indoor air contaminant concentrations on time scales of hours.
All have shown that these concentrations varied significantly with time - orders of magnitude on the timescale of a day or days.
\cite{holton_evaluation_2015,guo_vapor_2015,hosangadi_high-frequency_2017}. \par

In one instance the source of the variation was clearly established during the study; at the ASU house a field drain pipe (or “land drain”), which connected to a sewer system, was discovered beneath the house, and careful isolation of this source led to a clear conclusion that this preferential pathway significantly contributed to observed indoor air contaminant levels and their fluctuations\cite{guo_vapor_2015,guo_identification_2015}.
While in this case the issue of a contribution from a preferential pathway was clearly resolved, what it left open was a question of whether existence of such a preferential pathway to an area beneath a structure would always be expected to lead to large fluctuations in indoor air contaminant concentrations. \par

Similarly, a sewer pipe has recently been suggested to be a source of the contaminants found in the EPA Indianapolis house.
That site was also characterized by large indoor air contaminant concentration fluctuations\cite{mchugh_evidence_2017,u.s._environmental_protection_agency_assessment_2015}.
Sewer lines have been generally implicated as VI sources at several sites\cite{pennell_sewer_2013,mchugh_evidence_2017,roghani_occurrence_2018,riis_vapor_2010}.
A Danish study estimate that roughly 20\% of all VI sites in central Denmark involve significant sewer VI pathways\cite{nielsen_remediation_2017}.
Thus while the consideration of a role of possible sewer or other preferential pathways is now part of normal good practice in VI site investigation\cite{u.s._environmental_protection_agency_oswer_2015}, it is still not known whether the existence of such pathways automatically means that large temporal fluctuations are necessarily to be expected.
In some of these cited cases\cite{pennell_sewer_2013,riis_vapor_2010}, a sewer provided a pathway for direct entry of contaminant into the living space.
While potentially important in many cases, this scenario is not further considered here, where the focus is on pathways that deliver contaminant via the soil beneath a structure. \par

It is, however, now known that even absent a preferential pathway, there may be significant transient variation in indoor air contaminant concentrations at VI sites\cite{folkes_observed_2009,brenner_results_2010,johnston_spatiotemporal_2014}.
One example is a site at NAS North Island at which no preferential pathways have been identified.
Instead, a building at this site is characterized by significant temporal variations in indoor-outdoor pressure differential\cite{hosangadi_high-frequency_2017}.
It is believed that this is the origin of the observed indoor air contaminant concentration fluctuations at that site. \par

This paper investigates the sources of the temporal variation in indoor air contaminant concentrations in both the presence and absence of preferential pathways.
In this work, the latter scenarios are referred to as ”normal” VI scenarios, in which there is typically a groundwater source of the contaminant.
Specifically, we pose the question of just how much variation in indoor air contaminant concentration may be expected at  such normal  VI  sites vs. those characterized by preferential pathways.
The conditions required for preferential pathways to become significant contributors to temporal variations in indoor air contaminant concentrations are also explored, and the consequences for sampling strategies are also discussed.



\section{Methods}

\subsection{Statistical Analysis}

This paper heavily relies on statistical analysis of high resolution datasets from two well-studied VI sites, one near Hill AFB in Utah (called the ASU house) and another in Indianapolis, IN (simply called as such.)
Analysis is performed using the SciPy, NumPy, Pandas, and Seaborn Python packages.
Probability distributions of various parameters are constructed using the kernel density estimation (KDE) method\cite{altman_introduction_1992}, which is implemented in the SciPy package.

To begin to characterize transient behavior in indoor air contaminant concentrations, actual datasets are analyzed to establish common levels of variability at VI sites.
For this purpose, the datasets from the ASU house in Utah, the EPA Indianapolis site and North Island NAS were chosen for analysis.
This paper relies on statistical analysis of published field data, and readers are referred to the original works for details regarding data acquisition\cite{holton_evaluation_2015,guo_vapor_2015,holton_temporal_2013,hosangadi_high-frequency_2017,u.s._environmental_protection_agency_assessment_2015}. \par

The ASU house data were obtained over a period of a few years.
During part of this time, controlled pressure method (CPM) tests were being conducted, in which the house was underpressurized to an extent greater than that characterizing “normal” operation.
This caused greater than normal advective flow from the subsurface into the house, thus increasing VI potential\cite{mchugh_evaluation_2012,mchugh_recent_2017,holton_evaluation_2015}.
This period of CPM testing is considered separately from the otherwise ”natural” VI conditions in the analysis.
Likewise, the existence of a preferential pathway at the ASU house needs to be considered in examining the dataset, noting that during some of the testing, this pathway was deliberately cut off, resulting in what we have termed “normal” VI conditions in which the main source of contaminant was believed to be groundwater. \par

The NAS North Island dataset has not (as far as is known) been influenced by a preferential pathway, but the structure there was subject to large internal pressure fluctuations, much more extensive than those typically recorded at the ASU house during normal operations.
Additionally, the underlying soil at NAS North Island is sandy and more permeable than that at the ASU site, which, as will be shown, contributes to the indoor air contaminant concentrations being more sensitive to pressure fluctuations\cite{hosangadi_high-frequency_2017}. \par

Likewise, the Indianapolis site investigation spanned a number of years and periodically included the testing of a sub-slab depressurziation system (SSD).
The goal of the SSD testing was to mitigate the VI risk by drastically depressurizing the sub-slab area underneath the house, preventing the contaminants from entering the structure above.
Only the period before the installation of this system was considered in the present analysis.
It is likely a sewer line beneath the structure acted as a preferential pathway\cite{mchugh_evidence_2017}, however at no point was this preferential pathway removed, making it difficult to assess how significant the role of the preferential pathway was at this site.
Regardless of this it is of interest to consider the data from this site due to how extensive and complete the data collection was. \par

The typical variation in indoor air contaminant concentrations with time will first be considered below in the case of the ASU house during ”natural”, (i.e. non-CPM conditions), in the case of the NAS North Island site over the entire available dataset, and for the Indianapolis case we consider the variations before the installation of the SSD system.
The deviations in indoor air concentration from the mean TCE (and Chloroform and PCE at the Indianapolis site) values, as well as the indoor-outdoor pressure differentials associated with these concentrations were examined.
Both univariate and bivariate kernel density estimations (KDE) were constructed.
KDE is a technique that estimates the probability distribution of a random variable(s) by using multiple kernels, or weighting functions, and in this case, Gaussian kernels are used to create the KDEs.
This means that it is presumed that the variables of interest (i.e., indoor air contaminant concentrations and indoor-outdoor pressure differentials, as sampled) are normally distributed around mean values (and that there are statistical fluctuations associated with each sampling event).
In this instance, the scipy statistical package was used to construct the KDEs, assuming a bandwidth parameter determined by Scott's rule.
The distributions of the individual parameters and the relationship between them will be examined using the KDE method.


\section{Methods}

\subsection{Vapor Intrusion Model}

\begin{equation}
  V \frac{\partial c_\mathrm{in}}{\partial t} = n - c_\mathrm{in} A_e V \label{eq:indoor_air}
\end{equation}

\section{Results \& Discussion}

\subsection{Statistical Analysis of Field Data}

\begin{figure}[!h]
		\centering
    \caption{Distributions of indoor/outdoor pressure differences (top row), and its correlation with indoor air contaminant concentrations (middle rows), and air exchange rate (bottom row), at three different VI sites - ASU house, Indianapolis site, and North Island NAS.}
    \label{fig:pair_grid}
    \includegraphics[width=\textwidth]{simple_grid.pdf}
\end{figure}

Our discussion of examining the temporal variability of indoor air contaminant concentration begins by considering one of the most dynamic factors that influence VI - the indoor/outdoor pressure difference ($p_\mathrm{in/out}$).
In the top row of Figure \ref{fig:pair_grid}, the distribution of $p_\mathrm{in/out}$ at three difference VI sites, the ASU house, Indianapolis site, and North Island NAS, are plotted.
The distributions of $p_\mathrm{in/out}$ at these three site resemble normal-like distribution, with a mean of a few negativa Pa, indicating that the most probable state is that these sites are slightly depressized.
At the ASU house and the Indianapolis site, the range of $p_\mathrm{in/out}$ values are relatively similar, whereas the North Island NAS site has significalty larger span of values.
This is most likely due to the poor structural condition of the North Island site, making it much more suspectible to barometric pressure fluctations; the ASU house and Indianapolis site regular residential structure in good condition.
The periods when the preferential pathway at the ASU house was open and closed are considered separately.
% TODO: Add some sort of interesting conclusion

Now we move on to look at $p_\mathrm{in/out}$ is correlated with the indoor air contaminant concentration ($c_\mathrm{in}$).
The middle row of Figure \ref{fig:pair_grid} shows the log-10 values of $c_\mathrm{in}$ vs. $p_\mathrm{in/out}$, placed into evenly spaced bins, are plotted.
Each bin represents the mean $c_\mathrm{in}$ and the errorbars show the 95\% confidence interval. % TODO: Change this description once you've fixed the figure.
A linear regression curve is also fitted to the data, with the shaded part of the regression again indicating the 95\% confidence interval. % TODO: Make sure this is right.
At the ASU house, it is readibly apparent that the preferential pathway has a profound effect on how sensitive $c_\mathrm{in}$ is to $p_\mathrm{in/out}$, showing that the closing of the preferential pathway fundamentally changes the site.
Perhaps surprsingly, when the preferential pathway is closed $p_\mathrm{in/out}$ is not strongly correlated with $c_\mathrm{in}$.


It is clear that $p_\mathrm{in/out}$ sets a trend, i.e. increased depressurziation is associated with higher $c_\mathrm{in}$, but there is still considerable variability for a given $p_\mathrm{in/out}$.


$p_\mathrm{in/out}$ is important in determining the contaminant entry rate into a structure, but it also will partly determine the air exchange rate.
How significantly $p_\mathrm{in/out}$ determines $A_e$ varies significantly with site specific conditions, e.g. HVAC operation, if windows are open/closed, how well insulted the structure is etc.
$A_e$ is thus highly variate for a given $p_\mathrm{in/out}$ hypothesize that most of this variability may be explained by variation in air exchange rate ($A_e$).

Air exchange

This leads us to examine how $A_e$ is associated with


The difference in correlation of $c_\mathrm{in}$ vs. $p_\mathrm{in/out}$ for the "open" and "closed" periods, as well as the $p_\mathrm{in/out}$ is more evenly distributed around 0 Pa support with our choice of model, where we assume that the preferential pathway acts by enhancing the "advective potential" of a VI site through improving communication between the indoor and outdoor.


% ASU:
% Pressure distributions similar
% Yet drastically different c
% => clearly driven by PP
% Pressure relationship better when open than close
% => Enhanced advection, motivating our choice of model
% Poor relationship for closed,
% => Diffusion dominated transport
% Yet significant uncertainty and variance in indoor air?
% Look at the influence of Ae on this
% Pressure is related to Ae, but not determinant
% => still large variability in Ae for a given p, what effect does this have?

% Indianapolis & North Island
% Unforunately no Ae data here like ASU, cannot look at that influence
% Stronger correlation between p and c here
% Indianapolis had some type of PP, though less significant than at ASU
% North Island had very large pressure changes, may explain the stronger relationship


% Constant Ae simulations
% First consider only pressure influence (constant Ae)
% PP Open
% Clearly we can capture the trend when p < 0, but not well when p > 0
% When p > 0, diffusion limit, but still variability, why?
% => Indicate fluctuating Ae
% Uniform soil
% Removing gravel sub-base removes the "upswing"
% Permeable subbase region necessary to realize "advective potential"
% PP closed
% Kind of capture a "trend", but not variability
% Pressure not significant and/or advective transport too inhibited to be signifncant
% => Indicate fluctating Ae driver for variability here
% Main point:
% Pressure controls Ae and entry rate, but in closed and uniform cases, it is insufficnet to dramatically increase entry rate
% When open, this is no longer true
% Pressure also help control Ae, but there is great uncertainty
% This leads to the difficult to predict variability
% => Need to incorporate Ae fluctuations to understand variability

% New PP simulations with extreme Ae values for each pressure considered
% Now we capture most of the variability for open and closed
% Strong indicator that fluctating Ae is responsible for a significant portion of the variability
% Needs to be controlled better (hard)
% Or the span of values considered to give upper and lower limits on variability
% Use EPA handbook to estimate variability (pretty much can always be one OOM)

% This says nothing about at what speed or timescale change may occur
% We look at the difference between max/min across various timespans
% See that on daily to a week, absent a PP, little variability is expected
% => 24 hr average samples adequate
% => no need to sample more than a few days apart
% Across time, more of the extreme values of Ae and pressure have occured, kind of ergodically, giving rise to this increase in variability
% Remember p and Ae are more or less normally distributed, (Ae more skewed)
% => extreme values unlikely but do happen across time

% Problem with attenuation to subslab
% Spatial variability in subslab may be greater than temporal indoor variability
% ASU perfect example, here there is greater alpha_sb variability than c
% => May be more uncertain
% => False positive for indoor sources
% => But could also indicate presence of PP in subbase
% Much larger spatial variability in subbase with PP present than without
% (Show plot with 2-4 panes)
% 1./2. show ASU/Indianapolis alpha_sb
% 3./4. model alpha_sb distributions with and without PP




\begin{figure}[!h]
	\centering
	\begin{minipage}[c]{0.49\textwidth}
		\centering
    \caption{ }
    \label{fig:resampling}
    \includegraphics[width=\textwidth]{temporal_variability/resampling.pdf}
	\end{minipage}
	%\hspace{3cm}
	\begin{minipage}[c]{0.49\textwidth}
		\centering
    \caption{Simulated cases}
    \label{fig:land_drain_scenarios}
    \includegraphics[width=\textwidth]{land_drain_scenarios.pdf}
	\end{minipage}
\end{figure}



\subsubsection{Modeling Indoor Air Variability}

\begin{figure}[htb!]
  \caption{ }
  \label{fig:simulation_sampling}
  % ASU/North Island plot
  \begin{subfigure}{0.49\textwidth}
    \centering
    \caption{ }
    \label{fig:simulation_sampling_pp_open}
    \includegraphics[width=\textwidth]{simulation_prediction_span_open.pdf}
  \end{subfigure}
  % Indianapolis plot
  \begin{subfigure}{0.49\textwidth}
    \centering
    \caption{ }
    \label{fig:simulation_sampling_pp_closed}
    \includegraphics[width=\textwidth]{simulation_prediction_span_closed.pdf}
  \end{subfigure}
\end{figure}


\begin{acknowledgement}
  This project was supported by grant ES-201502 from the Strategic Environmental Research and Development Program and Environmental Security Technology Certification Program (SERDP-ESTCP).
\end{acknowledgement}

\bibliography{library}

\end{document}
