\documentclass[journal=esthag,manuscript=article]{achemso}

% additional packages
\usepackage[utf8]{inputenc}
\usepackage{amsmath}
\usepackage{booktabs}
\usepackage{subcaption}
\usepackage{tabularx}
\usepackage{array,multirow,graphicx}
\usepackage{comment}
\graphicspath{
  {./../figures/},
  {./../figures/air-exchange-rate/},
  {./../figures/kde-iacc-pressure/},
  {./../figures/preferential-pathway-sensitivity/},
  {./../figures/transient-response/}
}
\usepackage{lineno}
\linenumbers
% macros

% authors
\author{Jonathan G. V. Ström}
\affiliation[Brown University]{Brown University, School of Engineering, Providence, RI, USA}
\author{Yijun Yao}
\affiliation[Zhejiang University]{Zhejiang University, Hangzhou, China}
\author{Eric M. Suuberg}
\email{eric_suuberg@brown.edu}
\affiliation[Brown University]{Brown University, School of Engineering, Providence, RI, USA}

% title
\title{Transient Variability In Vapor Intrusion And The Factors That Influence It}

% keywords
\abbreviations{VI}
\keywords{VI, Preferential pathways, Temporal transient variability}

\begin{document}

\begin{abstract}
Temporal variability in indoor air contaminant concentrations at vapor intrusion (VI) sites has been a concern for some time.
We consider the source of the variability at VI sites located near Hill Air Force Base in Utah, an EPA experimental house in Indianapolis, IN and Naval Air Station North Island in California using statistical analysis methods and three-dimensional subsurface computational fluid dynamics modeling.
The results suggest that an order of magnitude variation in indoor air contaminant concentrations may be expected at ”normal” VI sites where preferential pathways do not play a role, whereas three or more orders of magnitude can be observed at sites characterized by preferential pathways.
A computational fluid dynamics modeling sensitivity analysis reveals that it is not only the presence of contaminant vapor in a preferential pathway that is required to see large observed variations, but there must also exist a permeable region beneath the structure.
Large temporal fluctuations in indoor air contaminant concentrations may be observed where no preferential pathways exist, but this requires a particular combination of large fluctuations in pressure driving force and high soil permeability.
\end{abstract}

\section{Introduction}

Long term vapor intrusion (VI) studies in both residential and larger commercial structures have raised concerns regarding significant observed transient behavior in indoor air contaminant concentrations\cite{u.s._environmental_protection_agency_oswer_2015,folkes_observed_2009,holton_temporal_2013,johnston_spatiotemporal_2014,hosangadi_high-frequency_2017,mchugh_recent_2017,u.s._environmental_protection_agency_assessment_2015}.
VI involves the migration of volatilizing contaminants from soil, groundwater or other subsurface sources into overlying structures. VI has been a recognized problem for some time, but many aspects remain poorly understood, particularly with respect to the causes of large temporal transients in indoor air concentrations.
There is uncertainty within the VI community regarding how to best develop sampling strategies to address this problem\cite{u.s._environmental_protection_agency_oswer_2015,holton_temporal_2013,johnson_integrated_2016}. \par

Results from a house operated by Arizona State University (ASU) near Hill AFB in Utah, an EPA experimental house in Indianapolis, IN and a large warehouse at the Naval Air Station (NAS) North Island, CA have all shown significant transient variations in indoor air contaminant concentrations.
All were outfitted with sampling and monitoring equipment that allowed tracking temporal variation in indoor air contaminant concentrations on time scales of hours.
All have shown that these concentrations varied significantly with time - orders of magnitude on the timescale of a day or days.
\cite{holton_evaluation_2015,guo_vapor_2015,hosangadi_high-frequency_2017}. \par

In one instance the source of the variation was clearly established during the study; at the ASU house a field drain pipe (or “land drain”), which connected to a sewer system, was discovered beneath the house, and careful isolation of this source led to a clear conclusion that this preferential pathway significantly contributed to observed indoor air contaminant levels and their fluctuations\cite{guo_vapor_2015,guo_identification_2015}.
While in this case the issue of a contribution from a preferential pathway was clearly resolved, what it left open was a question of whether existence of such a preferential pathway to an area beneath a structure would always be expected to lead to large fluctuations in indoor air contaminant concentrations. \par

Similarly, a sewer pipe has recently been suggested to be a source of the contaminants found in the EPA Indianapolis house.
That site was also characterized by large indoor air contaminant concentration fluctuations\cite{mchugh_evidence_2017,u.s._environmental_protection_agency_assessment_2015}.
Sewer lines have been generally implicated as VI sources at several sites\cite{pennell_sewer_2013,mchugh_evidence_2017,roghani_occurrence_2018,riis_vapor_2010}.
A Danish study estimate that roughly 20\% of all VI sites in central Denmark involve significant sewer VI pathways\cite{nielsen_remediation_2017}.
Thus while the consideration of a role of possible sewer or other preferential pathways is now part of normal good practice in VI site investigation\cite{u.s._environmental_protection_agency_oswer_2015}, it is still not known whether the existence of such pathways automatically means that large temporal fluctuations are necessarily to be expected.
In some of these cited cases\cite{pennell_sewer_2013,riis_vapor_2010}, a sewer provided a pathway for direct entry of contaminant into the living space.
While potentially important in many cases, this scenario is not further considered here, where the focus is on pathways that deliver contaminant via the soil beneath a structure. \par

It is, however, now known that even absent a preferential pathway, there may be significant transient variation in indoor air contaminant concentrations at VI sites\cite{folkes_observed_2009,brenner_results_2010,johnston_spatiotemporal_2014}.
One example is a site at NAS North Island at which no preferential pathways have been identified.
Instead, a building at this site is characterized by significant temporal variations in indoor-outdoor pressure differential\cite{hosangadi_high-frequency_2017}.
It is believed that this is the origin of the observed indoor air contaminant concentration fluctuations at that site. \par

This paper investigates the sources of the temporal variation in indoor air contaminant concentrations in both the presence and absence of preferential pathways.
In this work, the latter scenarios are referred to as ”normal” VI scenarios, in which there is typically a groundwater source of the contaminant.
Specifically, we pose the question of just how much variation in indoor air contaminant concentration may be expected at  such normal  VI  sites vs. those characterized by preferential pathways.
The conditions required for preferential pathways to become significant contributors to temporal variations in indoor air contaminant concentrations are also explored, and the consequences for sampling strategies are also discussed.

\section{Methods}

\subsection{Statistical Analysis Of Field Data}

To begin to characterize transient behavior in indoor air contaminant concentrations, actual datasets are analyzed to establish common levels of variability at VI sites.
For this purpose, the datasets from the ASU house in Utah, the EPA Indianapolis site and North Island NAS were chosen for analysis.
This paper relies on statistical analysis of published field data, and readers are referred to the original works for details regarding data acquisition\cite{holton_evaluation_2015,guo_vapor_2015,holton_temporal_2013,hosangadi_high-frequency_2017,u.s._environmental_protection_agency_assessment_2015}. \par

The ASU house data were obtained over a period of a few years.
During part of this time, controlled pressure method (CPM) tests were being conducted, in which the house was underpressurized to an extent greater than that characterizing “normal” operation.
This caused greater than normal advective flow from the subsurface into the house, thus increasing VI potential\cite{mchugh_evaluation_2012,mchugh_recent_2017,holton_evaluation_2015}.
This period of CPM testing is considered separately from the otherwise ”natural” VI conditions in the analysis.
Likewise, the existence of a preferential pathway at the ASU house needs to be considered in examining the dataset, noting that during some of the testing, this pathway was deliberately cut off, resulting in what we have termed “normal” VI conditions in which the main source of contaminant was believed to be groundwater. \par

The NAS North Island dataset has not (as far as is known) been influenced by a preferential pathway, but the structure there was subject to large internal pressure fluctuations, much more extensive than those typically recorded at the ASU house during normal operations.
Additionally, the underlying soil at NAS North Island is sandy and more permeable than that at the ASU site, which, as will be shown, contributes to the indoor air contaminant concentrations being more sensitive to pressure fluctuations\cite{hosangadi_high-frequency_2017}. \par

Likewise, the Indianapolis site investigation spanned a number of years and periodically included the testing of a sub-slab depressurziation system (SSD).
The goal of the SSD testing was to mitigate the VI risk by drastically depressurizing the sub-slab area underneath the house, preventing the contaminants from entering the structure above.
Only the period before the installation of this system was considered in the present analysis.
It is likely a sewer line beneath the structure acted as a preferential pathway\cite{mchugh_evidence_2017}, however at no point was this preferential pathway removed, making it difficult to assess how significant the role of the preferential pathway was at this site.
Regardless of this it is of interest to consider the data from this site due to how extensive and complete the data collection was. \par

The typical variation in indoor air contaminant concentrations with time will first be considered below in the case of the ASU house during ”natural”, (i.e. non-CPM conditions), in the case of the NAS North Island site over the entire available dataset, and for the Indianapolis case we consider the variations before the installation of the SSD system.
The deviations in indoor air concentration from the mean TCE (and Chloroform and PCE at the Indianapolis site) values, as well as the indoor-outdoor pressure differentials associated with these concentrations were examined.
Both univariate and bivariate kernel density estimations (KDE) were constructed.
KDE is a technique that estimates the probability distribution of a random variable(s) by using multiple kernels, or weighting functions, and in this case, Gaussian kernels are used to create the KDEs.
This means that it is presumed that the variables of interest (i.e., indoor air contaminant concentrations and indoor-outdoor pressure differentials, as sampled) are normally distributed around mean values (and that there are statistical fluctuations associated with each sampling event).
In this instance, the scipy statistical package was used to construct the KDEs, assuming a bandwidth parameter determined by Scott's rule.
The distributions of the individual parameters and the relationship between them will be examined using the KDE method.

\subsection{Modeling Work}
In addition to examining the actual field data, a previously described three -dimensional computational fluid dynamics model of a generic VI impacted house was used to elucidate certain aspects of  the processes.
This model was implemented in a finite element solver package, COMSOL Multiphysics.
In the present work, there has been an addition of a preferential pathway to the ”standard” model that has been described before in publications by this group\cite{shen_influence_2013,yao_investigating_2017,yao_three-dimensional_2017}.
As in the earlier studies, only the vadose zone soil domain is directly modeled.

The modeled structure is assumed to have a 10x10 m foundation footprint, with the bottom of the foundation slab lying 1 m below ground surface (bgs), simulating a house with a basement.
The indoor air space is modeled as a continuously stirred tank (CST)\cite{u.s._environmental_protection_agency_oswer_2015} and all of the contaminant entering the house is assumed to enter with soil gas through a 1 cm wide crack located between the foundation walls and the foundation slab around the perimeter of the house.
All of the contaminant leaving the indoor air space is assumed to do so via air exchange with the ambient.
The indoor control volume is assumed to consist of only of the basement, assumed as having a total volume of $300 \; \mathrm{m^3}$.
Clearly different assumptions could be made regarding the structural features and the size of the crack entry route, but for present purposes, this is unimportant as the intent is only to show for “typical” values what the influence of certain other features can be. \par

The modeled surrounding soil domain extends 5 meters from the perimeter of the house, and is assumed to consist of sandy clay (except as noted).
Directly beneath the foundation slab, there is assumed to be a 30 cm (one foot) thick gravel layer, except in certain cases where this sub-base material is assumed to be the same as the surrounding soil (termed  a ”uniform” soil scenario). \par

Where relevant The preferential pathway is modeled as a 10 cm (4”) pipe that exits into the gravel sub-base beneath the structure.
The air in the pipe is assumed to be contaminated with TCE at a vapor concentration equal to the vapor in equilibrium with the groundwater contaminant concentration below the structure, modified by a scaling factor $\chi$, allowing the contaminant concentration in the pipe to be parameterized. \par

The groundwater beneath the structure is assumed to be homogeneously contaminated with trichloroethylene (TCE) as a prototypical contaminant.
The groundwater itself is not modeled, as the bottom of the model domain is defined by the top of the water table.
The ground surface and the pipe are both assumed to be sources of air to the soil domain.
Both are assumed to be at reference atmospheric pressure. \par

Vapor transport in the soil is governed by Richard’s equation, a modified version  of Darcy’s Law, taking the variability of soil moisture in the vadose zone into account\cite{richards_capillary_1931}.
The van Genuchten equations are used to predict the soil moisture content and thus the effective permeability of the soil\cite{van_genuchten_closed-form_1980}.
The effective diffusivity of contaminant in soil is calculated using the Millington-Quirk model\cite{millington_permeability_1961}.
The transport of vapor contaminant in the soil is assumed to be governed by the advection-diffusion equation, in which either advection or diffusion may dominate depending upon position and particular circumstances.
The equations and the boundary conditions are given in Table \ref{tbl:eqns-bc-parameters}.

\begin{table}[htb!]
  \centering
  \caption{Governing equations, boundary conditions \& model input parameters. (See below for table of nomenclature).}
  \label{tbl:eqns-bc-parameters}
  \bigskip
  %%%%%%%%%%%%%%%%%%%%%%%%%%%%%%%%%%%%%%%%%%%%%%%%%%%%%%%%%%%%%%%%%%%%%%%%%%%%%%
  % Governing equations
  %%%%%%%%%%%%%%%%%%%%%%%%%%%%%%%%%%%%%%%%%%%%%%%%%%%%%%%%%%%%%%%%%%%%%%%%%%%%%%
  \subcaption{Governing equations}
  \begin{tabular}{l l}
    \toprule
    % Indoor air space equation
    Unsteady-CSTR                 & $V\frac{d u}{d t} = \int_{A_\mathrm{ck}} j_\mathrm{ck} dA - u A_e V$ \\
    % Richard's equation
    Richard's equation            & $\nabla \cdot \rho \Big( - \frac{\kappa_s}{\mu} k_r \nabla p \Big) = 0$ \\
    % Millington-Quirk equation
    Millington-Quirk              & $D_\mathrm{eff} = D_\mathrm{air}\frac{\theta_g^{10/3}}{\theta_t^2} + \frac{D_\mathrm{water}}{K_H} \frac{\theta_w^{10/3}}{\theta_t^2}$ \\
    % Advection-diffusion equation
    Advection-diffusion equation  & $\frac{\partial}{\partial t} \Big( \theta_w c_w + \theta_g c \Big) = \nabla (D_\mathrm{eff} \cdot \nabla c) - \vec{u} \cdot \nabla c$ \\
    % van Genuchten's equations
    \multirow{3}{*}{van Genuchten equations}     & $\mathrm{Se} = \frac{\theta_w - \theta_r}{\theta_t - \theta_r} = [1 + |\alpha z|^n]^{-m}$ \\
                                  & $\theta_g = \theta_t - \theta_w$ \\
                                  & $k_r = (1 - \mathrm{Se})^{l} [1 - (\mathrm{Se}^{-m})^m]^2$ \\
                                  & $m = 1 - 1/n$ \\
    \bottomrule
  \end{tabular}
  \bigskip
  %%%%%%%%%%%%%%%%%%%%%%%%%%%%%%%%%%%%%%%%%%%%%%%%%%%%%%%%%%%%%%%%%%%%%%%%%%%%%%
  % Boundary conditions
  %%%%%%%%%%%%%%%%%%%%%%%%%%%%%%%%%%%%%%%%%%%%%%%%%%%%%%%%%%%%%%%%%%%%%%%%%%%%%%
  \subcaption{Boundary conditions}
  \begin{tabular}{l l l}
    \toprule
    \textbf{Boundary}          & \textbf{Richard's equation}      &   \textbf{Advection-diffusion equation} \\
    % Foundation crack
    At foundation crack  & $p = p_\mathrm{in/out} \; \mathrm{(Pa)}$                            & $j_\mathrm{ck} = \frac{u c}{1 - \exp{(u L_\mathrm{slab}/D_\mathrm{air})}}$ \\
    % Groundwater source
    At groundwater source &  N/A & $c = c_\mathrm{gw} K_H \; \mathrm{(\mu g/m^3)}$ \\
    % Ground surface
    At ground surface      & $p = 0 \; \mathrm{(Pa)}$  & $c = 0 \; \mathrm{(\mu g/m^3)}$ \\
    % Preferential pathway
    Exit of preferential pathway  & $p = 0 \; \mathrm{(Pa)}$  & $c = c_\mathrm{gw} K_H \chi \; \mathrm{(\mu g/m^3)}$ \\
    \bottomrule
  \end{tabular}
  \bigskip
  %%%%%%%%%%%%%%%%%%%%%%%%%%%%%%%%%%%%%%%%%%%%%%%%%%%%%%%%%%%%%%%%%%%%%%%%%%%%%%
  % Soil input parameters
  %%%%%%%%%%%%%%%%%%%%%%%%%%%%%%%%%%%%%%%%%%%%%%%%%%%%%%%%%%%%%%%%%%%%%%%%%%%%%%
  \subcaption{Soil \& gravel properties\cite{dan_capillary_2012,abreu_conceptual_2012,u.s._environmental_protection_agency_userss_2004}}
  \begin{tabular}{l l l l l l l}
    \toprule
    % Descriptions
    Soil & $\text{Permeability} \; \mathrm{(m^2)}$  & $\mathrm{Density} \; \mathrm{(kg/m^3)}$  & $\theta_s$  & $\theta_r$  & $\alpha \; \mathrm{(1/m)}$  & $n$ \\
    % Gravel
    Gravel     & $1.3 \cdot 10^{-9}$   & 1680    & 0.42        & 0.005       & 100       & 3.1 \\
    % Sand
    Sand     & $9.9 \cdot 10^{-12}$  & 1430    & 0.38        & 0.053        & 3.5       & 3.2 \\
    % Sandy clay
    Sandy Clay    & $1.7 \cdot 10^{-14}$  & 1470    & 0.39        & 0.12        & 3.3       & 1.2 \\
    \bottomrule
  \end{tabular}
  \bigskip
  %%%%%%%%%%%%%%%%%%%%%%%%%%%%%%%%%%%%%%%%%%%%%%%%%%%%%%%%%%%%%%%%%%%%%%%%%%%%%%
  % TCE input parameters
  %%%%%%%%%%%%%%%%%%%%%%%%%%%%%%%%%%%%%%%%%%%%%%%%%%%%%%%%%%%%%%%%%%%%%%%%%%%%%%
  \subcaption{Trichloroethylene (diluted in air) properties\cite{abreu_conceptual_2012,u.s._environmental_protection_agency_userss_2004}}
  \begin{tabular}{l l l l l l}
    \toprule
    $D_\mathrm{air} \; \mathrm{(m^2/h)}$  & $D_\mathrm{water} \; \mathrm{(m^2/h)}$  & $\mathrm{Density} \; \mathrm{(kg/m^3)}$ & $\mathrm{Viscosity} \; \mathrm{(Pa \cdot s)}$  & $K_H$ & $M \; \mathrm{(g/mol)}$ \\
    $2.47 \cdot 10^{-2}$  & $3.67 \cdot 10^{-6}$  & 1.614 & $1.86 \cdot 10^{-5}$  & 0.403 & 131.39 \\
    \bottomrule
  \end{tabular}
  \bigskip
  %%%%%%%%%%%%%%%%%%%%%%%%%%%%%%%%%%%%%%%%%%%%%%%%%%%%%%%%%%%%%%%%%%%%%%%%%%%%%%
  % Building input parameters
  %%%%%%%%%%%%%%%%%%%%%%%%%%%%%%%%%%%%%%%%%%%%%%%%%%%%%%%%%%%%%%%%%%%%%%%%%%%%%%
  \subcaption{Building properties}
  \begin{tabular}{l l l}
    \toprule
    $V_\mathrm{base} \; \mathrm{(m^3)}$  & $L_\mathrm{slab} \; \mathrm{(cm)}$  & $A_e \; \mathrm{(1/hr)}$ \\
    %
    300  &  15  & 0.5 \\
    \bottomrule
  \end{tabular}
\end{table}

\subsection{Drivers For Indoor Air Contaminant Variability}
\subsubsection{The Role Of Building Depressurization/Pressurization}
% KDE figures
\begin{figure}[htb!]
  \caption{KDE analysis of IACC dependence on indoor/outdoor pressure difference at the ASU house and North Island site (\ref{fig:kde-asu-nas}) and the Indianapolis site (\ref{fig:kde-indianapolis}). p-values and Pearson's r-values shown for each dataset.}
  \label{fig:kde-analysis}
  % ASU/North Island plot
  \begin{subfigure}{\textwidth}
    \centering
    \caption{Period before and after the preferential pathway (PP) was shut at the ASU house considered separately; North Island dataset considered in its entirety.}
    \label{fig:kde-asu-nas}
    \includegraphics[height=0.4\textheight,keepaspectratio]{nas-asu-house-iacc-deviating.pdf}
  \end{subfigure}
  % Indianapolis plot
  \begin{subfigure}{\textwidth}
    \centering
    \caption{Chloroform and PCE considered at the Indianapolis duplex 422.}
    \label{fig:kde-indianapolis}
    \includegraphics[height=0.4\textheight,keepaspectratio]{indianapolis-422.pdf}
  \end{subfigure}
\end{figure}
% Introduction
Of the factors that influence indoor air contaminant concentration IACC in VI, pressure is one of the most dynamic ones, and the relationship between changes in indoor/outdoor pressure difference and IACC are examined in Figure \ref{fig:kde-analysis}.
Three well-studied VI sites are considered; the "ASU house" near Hill AFB in Utah, a building at North Island NAS in California and a duplex in Indianapolis.\par
% Univariate IACC description
The absolute IACCs at these sites vary significantly, therefore some means of comparing them to each other is necessary.
Additionally, the focus in this section is identifying the driver for variations in IACC, and consistent representation of the variations between the different sites must also be achieved.
Here variations on the scale of orders of magnitude is of interest, as these are of greatest concern in proper site IACC characterization.
To achieve these goals, log-10 deviation from the mean IACC (represented as $\mu$) within each dataset is calculated, e.g. $10\mu$ indicates an IACC that is an order of magnitude above the mean IACC for that dataset.
On the y-margins in Figures \ref{fig:kde-asu-nas} and \ref{fig:kde-indianapolis}, the univariate KDE distribution of deviation from the mean may be seen. \par

What these curves on the y-margins characterize are the distributions of sampled IACC at each site.
Naturally, the distributions are centered around the mean ($\mu = 1$).
In same caes, they look close to simple Gaussians in form (e.g. PCE, Tetrachloroethene at Indianapolis).
In most of the other cases, the distributions are curved and not of a simple nature, strongly suggesting that there might be several factors determinig the IACC. \par

% Univariate pressure description
The indoor/outdoor pressure differences are simply shown as reported, and their univariate KDE distributions are shown on the x-margins.
A negative pressure difference indicates that the building is underpressurized relative to ambient.\par
% Bivariate description
The relationship between the deviation from mean IACC, and indoor/outdoor pressure difference may be seen in the central portion of each figure, in the form of a bivariate KDE distribution.
The p-value and Pearson's r-value for each bivariate distribution are shown in the legend.\par
% Sub-figure description (details about sites)
In Figure \ref{fig:kde-asu-nas} the North Island and ASU house datasets are plotted, with blue representing the North Island site, and with red and green representing the ASU house.
The ASU house dataset is split into two different parts, the first (red) are data from the time period before the preferential pathway was discovered, the second (green) is from after the preferential pathway was sealed off.
At both the ASU house and North Island sites, TCE was the only contaminant considered.
The IACC of Chloroform and PCE in the 422 part of the Indianapolis duplex were considered in Figure \ref{fig:kde-indianapolis}, since they were particulary tracked.\par
% North Island discussion
From Figure \ref{fig:kde-asu-nas}, it is apparent that the three datasets differ significantly from each other.
The North Island site exhibits the greatest variation in both pressure difference and IACC, in fact, looking at the univariate pressure distribution - it is almost entirely flat and spanning from extremly high and low values.
A likely explation for this is the reportedly poor condition of the structure, rendering it highly susceptible to envivornmental influnces.\par

There is also a significant variation in IACC at the North Island site and this has a relativly strong link to pressure difference ($r = -0.71$), although there is clearly a distinct peak (around $0.5\mu$) that is associated with the smaller negative and positive pressure differences ($~-4 \leq \Delta p \leq 8$).
This demonstrates that at this particular site, the very large IACC variations (an order of magnitude or more) are primarily influenced by the unusually large pressure difference fluctations.
Likely, this relationship is created by the fact that the soil surrounding the structure is sandy and therefore relatively permeable - increasing the influence of pressure differences on inflow of contaminant from the soil.
However, despite this, pressure differences can only partly explain the variations observed at this site.\par
% ASU discussion
The two ASU datasets are not only significantly different from the North Island dataset, but also different from each other.
Considering the dataset before the preferential pathway was discovered (red), one can see that the IACC varies significantly (from $0.05 \mu$ to more than $50 \mu$, albeit the latter only rarely.)
This is more variation than was observed at North Island, and yet the observed distribution of pressure differences is narrow, mostly varying by a few Pa around zero (~$\pm 2$ Pa).
With $r = -0.50$, for this dataset, this suggests that pressure difference is not insignificant but still only a weak predictor for the variations in IACC.\par

The picture changes completely after the preferential pathway was closed off, where a similar distribution of pressure differences were observed, but with a signficant reduction in IACC variance to around $\pm 0.5 \mu$, resembling something of a log-normal distribution.
This clearly shows just how much variance the preferential pathway at this site contributed, and how critical it is to assess whether a preferential pathway is present at a particular site since this may lead to large uncertainties in determining relevant contaminant exposures.
It is also clear that in the absence of a preferential pathway, the pressure difference becomes even less important, with the p- and r-value indicating that pressure differences are insignficant in determining variation in IACC.\par
% Indianapolis discussion
The Indianapolis duplex also featured a preferential pathway, but despite this, there is significantly lower variance in IACC for both Chloroform and PCE\cite{mchugh_evidence_2017}.
The exact reason for this is not known, but considering that the preferential pathway at the Indianapolis site was a sanitary sewer, and at ASU the preferential pathway was a foundation drain system, it is not inconceivable that their dynamics are quite different.
Nevertheless, both PCE and Chloroform are more or less log-normally distributed, with (mostly) less than an order of magnitude variance around the mean, indicating that the varations in IACC are species independent.\par

The pressure difference distributions for the Indianapolis site exhibit some bimodal characteristics, with a larger variation than was observed at the ASU site.
This may explain why the r-values for both PCE and Chloroform, $r=-0.51$ and $r=-0.53$ respectively, are so similar to the ASU site before the preferential pathway there was closed; the larger pressure differences, which occured more often than ASU, increase contaminant entry rates to the extent that a preferential pathway was needed to "achieve" at the ASU site.\par
% Summary discussion
All of these data clearly suggest that there is significant variation in IACC that cannot easily be explained by variations in indoor/outdoor pressure difference - even with a preferential pathway.
It is also clear that a preferential pathway may contribute to unacceptable levels of IACC variations, and screening for preferential pathways should be part of any site investiation.
It may also be prudent, even in the absence of a preferential pathway, to add around a half to one order of magnitude margin of error to early screening measurements of indoor air, e.g. if a single sample is less than an order of magnitude below some target limit, it may be justified to perform subsequent sampling to better assess the relevant exposure.
These data also call the role of advection in VI into question, since "normal" pressure difference ranges are such a poor predictor of IACC variation.
This is the opposite of what one would expect if advection were the dominant transport mechanism for entry of contaminant into a structure, under steady-state conditions.\par

\subsubsection{The Role Of Air Exchange Rate}
% Air exchange rate plots
\begin{figure}[htb!]
  \caption{Comparison between the recorded  and the calculated TCE in indoor air at the ASU house, assuming constant air exchange rate. \ref{fig:asu-iacc-ae-overview} shows the TCE in indoor air across time as well as the exchange rate. \ref{fig:violin-iacc-aet} shows the distribution of these values for three periods.}
  \label{fig:ae-analysis}
  % "Overview" plot
  \begin{subfigure}{\textwidth}
    \caption{ }
    \label{fig:asu-iacc-ae-overview}
    \includegraphics[width=\textwidth]{tce_indoor_air_exchange_rate_impact.pdf}
  \end{subfigure}
  % Violinplot figure
  \begin{subfigure}{0.75\textwidth}
    \caption{ }
    \label{fig:violin-iacc-aet}
    \includegraphics[width=\textwidth]{violin-asu-iacc.pdf}
  \end{subfigure}
\end{figure}
Air exchange rate, which measures how often the interior air in a given building is exchanged with exterior air, is the main parameter characterizing contaminant expulsion, and part of another highly dynamic process.
Exploring the impact of fluctuations in air exchange rate is thus also important for the development of sampling strategies.
At the ASU house, the contaminant entry rate into the structure (called emission rate in that study) as well as the building air exchange rate were recorded across time.
This allows for exploration of the impact that fluctuating air exchange rate has on the final IACC by solving the CST equation, using the recorded contaminant entry rate as one input, and assuming a constant air exchange rate.
The rationale is to show the extent of variability in IACCs due to factors other than fluctuations in air exchange rates.
We have taken this factor out of consideration by below assuming a single constant exchange rate, characteristic of the whole period.

In Figure \ref{fig:asu-iacc-ae-overview}, the measured IACC and air exchange rate across time are shown, compared to the predicted IACC assuming instead a constant air exchange rate.
The median air exchange rate value for the pre-CPM period was chosen as this seemed to be the most representative value for this time period.
From Figure \ref{fig:asu-iacc-ae-overview} it may be seen quite clearly that for the non-CPM periods, the calculated IACC values assuming constant vs. actual fluctuating exchange rates are quite similar.
In other words, in this instance it is unlikely that the variations in IACC were driven by fluctuations in air exchange rate.

This is more clearly shown in Figure \ref{fig:violin-iacc-aet}, where KDEs of the IACC are constructed assuming the constant and fluctuating air exchange rate cases.
This figure shows the relative probability of being at a particular TCE level in indoor air concentration at the ASU house.
These are compared to each other for 1. the full measurement period, 2. the non-CPM periods and 3. the CPM period.
Looking at the full period, it does not appear as though the two cases of steady and fluctuating air exchange rate are comparable, the distribution functions for IACC are different and offset.
The curves to the different sides of the vertical line for the "full" measurement period represent the probability distributions for the indicated values of IACC.
It appears as though assuming a constant air exchange rate shifts the IACC probability distribution to higher values.
The reason for this are more apparent when considering the CPM and non-CPM periods separately.

Considering only the CPM period, assuming a constant air exchange rate does not influence the shape of the IACC distribution, but assuming constant exchange rate shifts the IACC to a higher values.
This is because the actual air exchange rate for this CPM period is significantly higher than the assumed median value, and much more contaminant is actually exchanged with the exterior.
The shape of the IACC distribution is not very different from that which takes the air exchange fluctuations into account, indicating that fluctuating exchange rate is not a major contributor to observed IACC variation for this period (the distinction being drawn between variation and absolute levels of IACC).

The non-CPM period is the condition under which field practitioners would normally collect samples at a VI site.
It is clear here that there is only a very minor difference made in assuming a constant vs. real fluctating air exchange rate, again indicating that accounting for the minor fluctuations of air exchange rate's impact on variations in IACC is not important.
\citeauthor{rackes_time-averaged_2016}\cite{rackes_time-averaged_2016} drew similar conclusions regarding the relationship between IACC and air exchange rate.

\subsection{The Influence of Preferential Pathways}

% Point is:
% If preferential pathways exist, then roughly 1-3 orders of magnitude variability are possible.

% Definition: A preferential pathway is a pathway that allows for contaminant vapor to enter a structure directly or close to the building - circumventing much of the resistance to transport in the soil.

% How do I deal with this? How do I know if preferential pathways may be an issue?

% Scenario: Preferential pathway directly connected to the building.
% - Identify sewer & plumbing fixtures.
% - Drains and sumps may be sources.

% Scenario: Preferential pathway in the near slab area.
% - Permeable sub-base necessary.
% - May have a limited “sphere of influence”.
% -- Limit likely not reached for a normal house.
% -- Limit more likely to be reached for larger structures.

% Relevant for both: Check local manhole/sewer for presence of contaminant - likely source for the preferential pathway.

\subsubsection{Conclusions From Steady-State Modeling}

\begin{figure}[htb!]
  \caption{Sensitivity analysis of IACC dependence on indoor/outdoor pressure difference for cases featuring a preferential pathway (\ref{fig:ss-sensitivity-analysis-pp}) and without a preferential pathway (\ref{fig:ss-sensitivity-analysis-no-pp}). Results compared with field data from ASU house.}
  \label{fig:ss-sensitivity-analysis}
  % "Overview" plot
  \begin{subfigure}{0.55\textwidth}
    \caption{preferential pathway present. Sensitivity to the presence of a gravel sub-base and contamination in the preferential pathway considered.}
    \label{fig:ss-sensitivity-analysis-pp}
    \includegraphics[width=\textwidth]{yes-pp.pdf}
  \end{subfigure}
  % Violinplot figure
  \begin{subfigure}{0.55\textwidth}
    \caption{No preferential pathway present. Sensitivitiy to the presence of a gravel sub-base considered.}
    \label{fig:ss-sensitivity-analysis-no-pp}
    \includegraphics[width=\textwidth]{no-pp.pdf}
  \end{subfigure}
\end{figure}

Clearly, preferential pathways pose a major problem for site investigations, in particular because they may be difficult or even impossible to anticipate or uncover.
Even at a well studied site like the ASU house it took years to discover the preferential pathway.
Therefore, there is a need to consider other indications that a preferential pathway is present or a potential issue.\par

To investigate this, inspiration is taken from the ASU house, where there clearly was a preferential pathway containing contaminant vapor, which exited into a gravel (permeable) sub-base.
The impact of these features is simulated by examining the IACC at different indoor/outdoor pressure differences for each combination of cases.
Instead of representing IACC as absolute concentration, we now non-dimensionalize with respect to the vapor in equilibrium with the groundwater contaminant concentration.
This non-dimensionalized property is commonly called the attenuation factor and is denoted by $\alpha_\mathrm{gw}$.
The result of a steady-state sensitivity analysis relative to several factors is seen in Figure \ref{fig:ss-sensitivity-analysis}.\par

Figure \ref{fig:ss-sensitivity-analysis-pp} deal with cases where a preferential pathway is present and, the impact of having a gravel vs. soil sub-base and vapor contaminant vs. clean air in the preferential pathway ($\chi=1$ and $\chi=0$ respectively) are considered.
Figure \ref{fig:ss-sensitivity-analysis-no-pp} considers the cases absent a preferential pathway, for reference, but still considers the impact of a gravel vs. soil sub-base.
To give perspective to these results, the IACC at the ASU house (as $\alpha_\mathrm{gw}$) vs. indoor/outdoor pressure difference are also plotted.
But the reader is reminded that the model was examined for a structure of the same scale as the ASU house, but which is not an exact representation of it.
Only data from period before the preferential pathway was discovered, and after the preferential pathway was closed are considered in each respective figure (\ref{fig:ss-sensitivity-analysis-pp} and \ref{fig:ss-sensitivity-analysis-no-pp}).
To improve visibility of the ASU data, it's average $\alpha_\mathrm{gw}$ values were aggregated into up to 40 evenly spaced points.\par

Begining with Figure \ref{fig:ss-sensitivity-analysis-pp} it is immediately clear that absent a gravel sub-base, the impact of a preferential pathway is minimal, as comparing the orange and blue lines shows.
Since in the model, it is assumed that the soil type is sandy clay, which is relatively impermeable, there is simply too much resistance to advection in the soil for changes in pressurization to matter.
Additionally, whether the preferential pathway is filled with contaminant vapors or not does not matter; without a permeable sub-base, the transport of contaminant vapor in the sub-base is too slow to allow an advective entry path to contribute with.\par

When a gravel sub-base is present, and the preferential pathway is filled with contaminant vapors ($\chi=1$).
Tthe preferential pathway has a major impact on $\alpha_\mathrm{gw}$, spanning more than four orders of magnitude.
This is explained by the increased role of advective transport under these conditions.
The gravel sub-base allows vapor to readily flow.
When the building is underpressurized under these conditions, a significant amount of contaminant vapor is pulled in from the preferential pathway, leading to the large increase of $\alpha_\mathrm{gw}$.\par

On the other hand, when advection is stopped by overpressurization, most of the contaminant entering the building does so only through diffusion.
This also explains why there is no difference whether the preferential pathway brings only clean air or contaminated air (red and green) when the building is overpressurized.
These two cases differ signficantly from one another when the building is underpressurized, obviously because if there is no additional contaminant being supplied by the preferential pathway, the contaminant comes only from what diffuses from the contaminted groundwater source.\par

It is interesting to note that $\alpha_\mathrm{gw}$ is higher with an uncontaminted preferential pathway (green) entering a gravel sub-base than the cases without a gravel sub-base.
One could expect that pulling so much clean air into the sub-base region might significantly dilute the contaminant vapors and if anything cause a decrease in $\alpha_\mathrm{gw}$ (relative to blue and orange).
However, the preferential pathway is small relative to the dimensions of the sub-base (10 cm diameter pipe), and its impact is only on the contaminant concentration in a limited region of the sub-base.
This gravel sub-base still allows for higher advection throughout the entire sub-base region, leading to a $\alpha_\mathrm{gw}$ plateau that is limited by contaminant transport from the groundwater source, and any small amount of dilution from the preferential pathway is negligible.\par

These case most resembling the ASU site (with an open preferential pathway) is the red case, and the trend agrees with the field data fairly well, especially for small under-/overpressurization values.
The most significant deviations occur as the under-/overpressurization becomes larger.
Again, the model is not intended to replicate the ASU data perfectly, but rather investigate the factors make a preferential pathway the most impactful in a general sense, and perfect simulation is not to be expected.
This is especially reflected in the choice of vapor contaminant concentration in the preferential pathway, with it either being equal to the groundwater contaminant vapor concentration ($\chi=1$) or clean ($\chi=0$), neither of which is likely to be perfectly true at the ASU site.
The idea with picking $\chi=1$ and $\chi=0$ is to give a span of $\alpha_\mathrm{gw}$ values within which a preferential pathway may fall.
This helps explain why $\alpha_\mathrm{gw}$ is overpredicted when the structure is increasingly underpressurized - the contaminant vapor in the preferential pathway at the ASU house was lower than the groundwater contaminant vapor concentration.
The simulation are able to capture the general trends well.\par

In Figure \ref{fig:ss-sensitivity-analysis-no-pp} cases where a preferential pathway is absent are considered, i.e. "normal" VI scenarios, which may be considered reference scenarios.
Here the only issues considered is whether there is a gravel or soil sub-base.
It is apparent that without a preferential pathway, the indoor/outdoor pressure difference changes do not cause any signficant change in $\alpha_\mathrm{gw}$, due to the resistance to advective transport in the surrounding soil - showing the importance of the preferential pathway for increasing advection.
Furthermore, the gravel sub-base does increase the overall $\alpha_\mathrm{gw}$, partly due to slightly higher advective potential but also due to the contaminant vapors being able to more easily diffuse in the more permeable sub-base.
The trend in and even the value sof the data from the ASU house after the preferential pathway had been closed agree well with these model predictions, clearly again demonstrating the role of preferential pathway in increasing advective potential at a site.
It should also be stated that if the surrounding soil was more permeable, e.g. sand, there would be more of a dependence of $\alpha_\mathrm{gw}$ on pressurization, as seen in North Island.\par

Based on the simulated cases in Figure \ref{fig:ss-sensitivity-analysis} it can be concluded that a preferential pathway can be a signficant contributor to VI, but only when particular conditions are met.
\begin{enumerate}
  \item The preferential pathway has to supply additional vapor contaminant.
  \item There must be a source of increased advective potential, which may be due to the presence of the preferential pathway itself, or other site features, or simply due to more permeable surrounding soil.
  \item A permeable sub-base must exist to realize increased advective potential.
\end{enumerate}
In the absense of any of these conditions, a preferential pathway is unlikely to be a signficant contributor to VI.
Strategies for screening for preferential pathway are discussed by \citeauthor{nielsen_remediation_2017}\cite{nielsen_remediation_2017}. \par


\begin{comment}

Clearly, PPs pose a major problem for site investigations, in particular because they may be difficult or even impossible to anticipate or uncover; even at a well studied site like the ASU house it took years to discover the PP.
Therefore, there is a need to consider other indications that a PP is present.

% Sensitivity analysis plot showing when a PP is relevant
\begin{figure}[htb!]
  \caption{Attenuation factor relative to contaminant vapor in equilibrium with groundwater as a function of indoor-outdoor pressure difference. The effects of a preferential pathway, that is either contaminated or uncontaminated as well as that of having a gravel sub-base vs. uniform sandy clay soil are considered.}
  \label{fig:model_results_sensitivity_analysis}
  % Upper left
  \begin{subfigure}[t]{0.45\textwidth}
    \caption{ }
    \label{fig:model_results_sensitivity_analysis_0}
    \includegraphics[width=\textwidth]{ss_pp_sensitivity_analysis_0.png}
  \end{subfigure}
  % Upper right
  \begin{subfigure}[t]{0.45\textwidth}
    \caption{ }
    \label{fig:model_results_sensitivity_analysis_1}
    \includegraphics[width=\textwidth]{ss_pp_sensitivity_analysis_1.png}
  \end{subfigure}
  % Middle left
  \begin{subfigure}[t]{0.45\textwidth}
    \caption{ }
    \label{fig:model_results_sensitivity_analysis_2}
    \includegraphics[width=\textwidth]{ss_pp_sensitivity_analysis_2.png}
  \end{subfigure}
  % Middle right
  \begin{subfigure}[t]{0.45\textwidth}
    \caption{ }
    \label{fig:model_results_sensitivity_analysis_3}
    \includegraphics[width=\textwidth]{ss_pp_sensitivity_analysis_3.png}
  \end{subfigure}
  % Lower left
  \begin{subfigure}[t]{0.45\textwidth}
    \caption{ }
    \label{fig:model_results_sensitivity_analysis_4}
    \includegraphics[width=\textwidth]{ss_pp_sensitivity_analysis_4.png}
  \end{subfigure}
  % Lower right
  \begin{subfigure}[t]{0.45\textwidth}
    \caption{ }
    \label{fig:model_results_sensitivity_analysis_5}
    \includegraphics[width=\textwidth]{ss_pp_sensitivity_analysis_5.png}
  \end{subfigure}
\end{figure}

One way has been suggested by the situation at the ASU house, where there clearly was a PP containing contaminant vapor, and the PP exited into a gravel (permeable) sub-base.
This situation can be examined using the VI model introduced earlier by examining how IACC is impacted by various factors over a range of indoor-outdoor pressure difference values.
Instead of representing IACC as absolute concentration, we now non-dimensionalize with respect to the vapor in equilibrium with the groundwater contaminant concentration.
This non-dimensionalized property is commonly called the attenuation factor and is denoted by $\alpha_\mathrm{gw}$.
The result of a sensitivity analysis relative to several factors is seen in Figure \ref{fig:model_results_sensitivity_analysis}.
These examine scenarios in which the indoor-outdoor pressure difference is a driver for VI.

In this figure, all the graphs in the left column show results for a scenario, in which the house has a gravel sub-base, with the remaining surrounding soil assumed to be sandy clay.
All the right column graphs features only sandy clay soil, directly in contact with the slab.
Figure \ref{fig:model_results_sensitivity_analysis_0} considers an uncontaminated PP ($\chi = 0$), i.e. the PP delivers clean air to beneath the slab.
In \ref{fig:model_results_sensitivity_analysis_1} there is no PP present at all, giving a reference state for a "normal" VI scenario.
\ref{fig:model_results_sensitivity_analysis_2} features a contaminated PP ($\chi = 1$), i.e. the PP delivers contaminant vapor at a concentration characteristic of underlying groundwater.

In terms of conditions that may lead to significant temporal transients in $\alpha_\mathrm{gw}$, driven by indoor-outdoor pressure difference, most of the cases in this analysis show negative results.
There are only two combinations of factors where significant changes in $\alpha_\mathrm{gw}$ accompany the changes in indoor-outdoor pressure difference, both involving the permeable (gravel) sub-base into which a PP enters.
In \ref{fig:model_results_sensitivity_analysis_2}, the PP is filled with contaminant vapor whereas figure \ref{fig:model_results_sensitivity_analysis} involves the PP only delivering "clean" air to the gravel sub-base.
When the building is overpressurized (positive indoor-outdoor pressure difference), the two cases exhibit similar behavior, i.e. $\alpha_\mathrm{gw}$ is low as expected.

When the building is underpressurized, the uncontaminated PP case involves barely any increase in $\alpha_\mathrm{gw}$ while there is a very significant increase in $\alpha_\mathrm{gw}$ in the contaminated PP case.
For the contaminated PP scenario, this is quite easy to understand, as the extra air flow provided by the PP into the gravel sub-base very effectively disperses contaminant vapor across the sub-base and the contaminant entry rate is increased.

When the PP is provides clean air, the increase in air flow into the sub-base with depressurization is the same.
However, the resulting increase in air entry rate does not bring in more contaminant as clean air is being drawn in.
This means that most of the contaminant entry rate is still only controlled by the usual process of diffusion through the soil, giving a result that is actually similar to that for the "no PP and with gravel sub-base" scenario.
Thus, for a PP to be a significant contributor to contaminant entry at a VI site the conditions such as featured in the bottom left case are required, i.e. a permeable sub-base and the PP must be a source of contaminant.

Various techniques for determining the potential for sewer gas entry into a house are described by \citeauthor{nielsen_remediation_2017-1}\cite{nielsen_remediation_2017-1}.
Collecting contaminant vapor samples from nearby manholes is also recommended, however one should be aware that contaminant vapors can potentially travel very long distances in a sewer system\cite{mchugh_evidence_2017,roghani_occurrence_2018,riis_vapor_2010}.

% Models specific to ASU & North Island
\begin{figure}[htb!]
  \caption{Attenuation factor vs. indoor-outdoor pressure difference when considering VI scenarios similar to the ASU house and the North Island NAS site (\ref{fig:ss_asu} and \ref{fig:ss_north_island} respectively).}
  \label{fig:model_results_asu_north_island}
  % ASU
  \begin{subfigure}[t]{0.45\textwidth}
    \caption{ }
    \label{fig:ss_asu}
    \includegraphics[width=\textwidth]{ss_asu.png}
  \end{subfigure}
  % North Island
  \begin{subfigure}[t]{0.45\textwidth}
    \caption{ }
    \label{fig:ss_north_island}
    \includegraphics[width=\textwidth]{ss_north_island.png}
  \end{subfigure}
\end{figure}

At the ASU house, after the PP was closed, there was still roughly an order of magnitude variation in IACC, the reasons for which Figure \ref{fig:model_results_sensitivity_analysis} does not address.
Likewise, there was more than two orders of magnitude observed at the North Island site over the indoor-outdoor pressure difference ranges simulated despite there not being a PP present.
Therefore, it is of interest to examine these sites a bit further using the model.

The foundation crack at the ASU house seems by all accounts to be quite small, roughly 180x1 cm and located close to the PP\cite{guo_vapor_2015}.
This crack is smaller than the one assumed in the model used in Figure \ref{fig:model_results_sensitivity_analysis}.
The influence of indoor-outdoor pressure difference is explored for a case in which a smaller crack area is assumed.
We also consider the effects of having a significantly lower air exchange rate, 0.1 vs. the regular 0.5 per hour.

The results of the case shown in Figure \ref{fig:model_results_sensitivity_analysis_2} is shown in Figure \ref{fig:ss_asu}.
Even though there is an increase in $\alpha_\mathrm{gw}$ with increased underpressurization, it is no where near the roughly one order of magnitude recorded at the ASU house after the PP was closed.
So while crack dimensions can clearly influence $\alpha_\mathrm{gw}$, they cannot be used to infer a bigger role for pressure difference in temporal variations in $\alpha_\mathrm{gw}$ or IACC.
The influence of air exchange rate on $\alpha_\mathrm{gw}$ predictions is also shown in Figure \ref{fig:model_results_asu_north_island}.
Clearly, the value of air exchange rate at steady-state conditions will only influence absolute concentrations in the structure.
Ultimately, the reason for the transient variability in IACC during the post-CPM period at the ASU house remains elusive, as a Pearson's r analysis between IACC and various factors do not yield any explanation.

In the case of the North Island site, variations in IACC are clearly not related to the existence of a PP.
The soil underlying the North Island site is more sand-like\cite{hosangadi_high-frequency_2017}.
The significance of this is that sand soil is much more permeable than sandy clay soil, which means that $\alpha_\mathrm{gw}$ will be much more sensitive to changes in indoor-outdoor pressure difference.
Thus, another "normal" VI scenario was run but this time with sand soil instead of sandy clay, the results of which may be seen in Figure \ref{fig:ss_north_island}.
Under these conditions, $\alpha_\mathrm{gw}$ changes significantly with pressurization, spanning roughly two orders of magnitude.
This is consistent with what was observed at North Island, as there the span of IACC values was roughly the same across this range of indoor-outdoor pressure difference values.
This simulation, together with the Pearson $r = -0.64$ for North Island in Figure \ref{fig:kde-asu-nas} suggest that under conditions where the soil around a structure is relatively permeable, indoor-outdoor pressure difference can be a significant driving forces for transient changes in IACC.
Therefore, recording the indoor-outdoor pressure difference at a site (in particular if the soil is permeable) may offer insight into the potential for transient changes in IACC (some good methods for doing this are suggested by \citeauthor{nielsen_remediation_2017-1}\cite{nielsen_remediation_2017-1}).

% Transient part
\subsubsection{Transient Modeling}

Analyzing the impact of a PP using steady-state simulations such as those in Figures \ref{fig:model_results_sensitivity_analysis} and does not revealing about how quickly IACC can change in the presence of a PP.
A transient simulation of a VI scenario that is characterized by a PP has been performed, where only the indoor-outdoor pressure difference was temporally changed.
The pressure difference is changed in a sinusoidal fashion, defined by \eqref{eq:p_transient},
\begin{equation}
  \label{eq:p_transient}
  p_\mathrm{in/out} = 10 \sin{(2 \pi t)} - 0.5
\end{equation}
where $t$ is given in days and $p_\mathrm{in/out}$ is the indoor-outdoor pressure difference in Pa.
Figure \ref{fig:transient_simulation} shows the results of this simulation.

\begin{figure}
  \caption{Transient response of TCE in indoor air (as attenuation factor) and in the surrounding soil \& gravel sub-base for a PP VI scenario subject to sinusoidal indoor-outdoor pressure fluctuations.
  \ref{fig:alpha_response} shows the attenuation factor and indoor-outdoor pressure across time.
  Figures \ref{fig:soil_30_hr} \& \ref{fig:soil_42_hr} show the contaminant concentration (normalized to the contaminant vapor in equilibrium with groundwater) in the surrounding soil \& gravel sub-base at different depths and times.}
  \label{fig:transient_simulation}
  % Transient response figure
  \begin{subfigure}[t]{0.75\textwidth}
    \caption{ }
    \label{fig:alpha_response}
    \includegraphics[width=\textwidth]{alpha_transient.png}
  \end{subfigure}
  % Soil t = 30 hr ...
  \begin{subfigure}[c]{0.45\textwidth}
    \caption{t = 30 hr}
    \label{fig:soil_30_hr}
    \includegraphics[width=\textwidth]{tri_layer_soil_concs1.png}
  \end{subfigure}
  % Soil t = 42 hr ...
  \begin{subfigure}[c]{0.45\textwidth}
    \caption{t = 42 hr}
    \label{fig:soil_42_hr}
    \includegraphics[width=\textwidth]{tri_layer_soil_concs2.png}
  \end{subfigure}
  % Color map
  \begin{subfigure}[c]{0.05\textwidth}
    \phantomcaption{ }
    %\label{fig:cmap}
    \includegraphics[width=\textwidth]{cmap.png}
  \end{subfigure}
\end{figure}

Figure \ref{fig:alpha_response} shows the IACC response, given as attenuation factor relative to contaminant vapor in equilibrium with groundwater, and the indoor-outdoor pressure difference with time.
What is apparent is that $\alpha_\mathrm{gw}$ closely follows the indoor-outdoor pressure difference, with the max./min. $\alpha_\mathrm{gw}$ being reached roughly 0.5 hours after the indoor-outdoor pressure difference reaches its max./min..
The reader should be aware that the assumption that the indoor air space can be modeled as a CST overestimates how dynamic this process is as a real house probably would not be perfectly mixed on as a short time scale as this.
Regardless, this suggests that VI sites characterized by PPs are highly dynamic and IACC's may change quickly in response to changing indoor-outdoor pressure differences, consistent with what was observed at the ASU house\cite{holton_temporal_2013}.
Again, the dynamics of the indoor-outdoor pressure difference at a VI site can offer insight into the potential for transient behavior of IACC.

Figures \ref{fig:soil_30_hr} and \ref{fig:soil_42_hr} show the contaminant concentration (normalized to the contaminant vapor in equilibrium with groundwater) in the soil and gravel sub-base at three different depths at times of 30 and 42 hr respectively corresponding to roughly the maximum in overpressurization and depressurization respectively.
The top layer is 3 m above groundwater and immediately under the house foundation (indicated by the black square).
The middle and bottom layers are 2 and 1 m above groundwater respectively.
The black streamlines show the contaminant transport flow paths.
The region closest to the PP is expanded in the circular insets.

The impact of the PP on the contaminant vapor concentration in the subsurface is apparent in these figures, as one easily sees how (especially) the gravel sub-base is filled up.
The contaminant vapor concentration is the highest closest to the PP (which is located at the edge of the gravel sub-base) and decreases with increased distance from the PP.
Note that a very small amount of contaminant vapor leaves the gravel sub-base, as the sandy clay has a relatively large resistance to transport compared with gravel.
This makes it difficult to detect the contribution of the PP outside the foundation footprint.
However, this may not always be true, as vapor contaminant from the PP would disperse more easily into more permeable soils.

It should also be noted that the vapor contaminant emanating from the PP in this model is relatively high concentration; the vapor contaminant concentration is equal to that in equilibrium with the groundwater source.
This does not always seem to be the case, as VI sites that have uncovered PPs find that the vapor contaminant emanating from the PP can be lower than that\cite{guo_vapor_2015}.
Therefore, it may be much more difficult than it may appear in Figures \ref{fig:soil_30_hr} and \ref{fig:soil_42_hr} to localize a vapor contaminant hotspot as a way to determining the presence of a PP at a particular VI site.

The insert circles in figures \ref{fig:soil_30_hr} and \ref{fig:soil_42_hr} also show that for the simulated time period, only a small amount of the contaminant enters the structure (as evident by the blue low concentration zone around the perimeter), yet a very large increase in $\alpha_\mathrm{gw}$ occurs.
The concentration in the rest of the sub-base is unchanged.
This suggests that once a PP has deposited contaminant underneath a structure, it may persist for a long time period, as pointed out by \citeauthor{guo_vapor_2015}\cite{guo_vapor_2015}.

\end{comment}

\begin{acknowledgement}
This project was supported by grant ES-201502 from the Strategic Environmental Research and Development Program and Environmental Security Technology Certification Program (SERDP-ESTCP).
\end{acknowledgement}

\begin{table}[htb!]
  \caption{Abbreviations \& symbols}
  \begin{tabular}{l l}
  \toprule
  \textbf{Abbreviation or symbol}     & \textbf{Explanation} \\
  % General abbreviations
  VI                                  & Vapor intrusion \\
  PP                                  & Preferential pathway \\
  $L_\mathrm{slab}$                   & Thickness of the foundation concrete slab \\
  % Richard's equation
  $p$                                 & Pressure \\
  % Unsteady-cstr
  $u$                                 & TCE in indoor air concentration \\
  $c$                                 & TCE in soil gas concentration \\
  % Diffusion stuff
  $D_\mathrm{air}$                    & Diffusion of TCE in air \\
  $D_\mathrm{water}$                  & Diffusion of TCE in water \\
  $D_\mathrm{eff}$                    & Effective Diffusion of TCE in soil \\
  $K_H$                               & Henry's Law constant \\
  % van Genuchten stuff
  $\theta_g, \theta_w\;\&\;\theta_s$  & Vapor, water filled \& saturated soil porosity \\
  $\theta_r$                          & Residual moisture porosity \\
  $\alpha,\; l,\; n\;\&\; m$          & van Genuchten parameters \\
  $\mathrm{Se}$                       & Soil moisture saturation \\
  $k_r$                               & Relative permeability \\
  $\alpha_\mathrm{gw}$                & Attenuation factor \\
  \bottomrule
  \end{tabular}
\end{table}

\bibliography{library}

\end{document}
